\documentclass[a4paper,14pt]{extarticle}

% PACKAGES %

% Language
\usepackage[T2A]{fontenc}
\usepackage[utf8]{inputenc}
\usepackage[russian,english]{babel}

% Math

\usepackage{amsmath}
\usepackage{amssymb}
\usepackage{amsfonts}
\usepackage{amsthm}
\usepackage{mathtools}
\usepackage{mathrsfs}
\usepackage{tikz-cd}

% Other

\usepackage{csquotes}
\usepackage{hyperref}
\usepackage[
    backend=biber,
    style=alphabetic,
    sorting=ynt
]{biblatex}

\usepackage{comment}

% PACKAGES OPTIONS %

\addbibresource{references.bib}

\theoremstyle{definition}
\newtheorem{definition}{Определение}[section]
\newtheorem{denotation}[definition]{Обозначение}
\newtheorem{theorem}{Теорема}[section]
\newtheorem{claim}[theorem]{Утверждение}
\newtheorem{guess}{Предположение}
\newtheorem{remark}{Замечание}[section]
\newtheorem*{question}{Вопрос}
\renewcommand\qedsymbol{$\blacksquare$}

\hypersetup{
    colorlinks=true,
    linkcolor=[rgb]{0.212,0.565,0.753},
    citecolor=[rgb]{0,0.506,0.541},
    urlcolor=[rgb]{0,0.275,0.212},
}

% COMMANDS %

\newcommand{\Ab}{\mathbf{Ab}}
\newcommand{\Cart}{\mathrm{Cart}}
\newcommand{\id}{\mathrm{id}}
\newcommand{\ord}{\mathrm{ord}}
\newcommand{\Sets}{\mathbf{Sets}}

\newcommand{\End}[1]{\mathrm{End}\left(#1\right)}
\newcommand{\fchar}[1]{\mathrm{char}\left(#1\right)}
\newcommand{\Compl}[1]{\mathbf{Compl}_{#1}}
\newcommand{\forget}[1]{\phi_{#1}}
\newcommand{\Ker}[1]{\mathrm{Ker}\left(#1\right)}
\newcommand{\Lie}[1]{\mathrm{Lie}\left(#1\right)}
\newcommand{\Mod}[1]{\mathbf{Mod}_{#1}}
\newcommand{\Nil}[1]{\mathbf{Nil}_{#1}}
\newcommand{\rank}[1]{\mathrm{rank}\left(#1\right)}
\newcommand{\Spf}[1]{\mathrm{Spf}\left(#1\right)}

\newcommand{\Hom}[2]{\mathrm{Hom}\left(#1, #2\right)}
\newcommand{\GL}[2]{\mathrm{GL}_{#1}\left(#2\right)}

\newcommand{\modhom}[3]{{#1}^{{\scriptscriptstyle #2, #3 }}}
\newcommand{\nat}[3]{{#1}^{{\scriptscriptstyle #2, #3 }}}

\newcommand{\Beta}{\mathrm{B}}

\newcommand{\bC}{\mathbb{C}}
\newcommand{\bF}{\mathbb{F}}
\newcommand{\bG}{\mathbb{G}}
\newcommand{\bN}{\mathbb{N}}
\newcommand{\bQ}{\mathbb{Q}}
\newcommand{\bZ}{\mathbb{Z}}

\newcommand{\hbG}{\hat{\bG}}

\newcommand{\fA}{\mathfrak{A}}
\newcommand{\fM}{\mathfrak{M}}
\newcommand{\fO}{\mathfrak{O}}
\newcommand{\fa}{\mathfrak{a}}
\newcommand{\fb}{\mathfrak{b}}
\newcommand{\fc}{\mathfrak{c}}
\newcommand{\fp}{\mathfrak{p}}

\newcommand{\oH}[1]{\left\langle #1 \right\rangle}
\newcommand{\eH}[1]{\left[ #1 \right]}
\newcommand{\oF}{\mathbf{f}}
\newcommand{\eF}{\mathcal{F}}
\newcommand{\oV}{\mathbf{V}}
\newcommand{\eV}{\mathcal{V}}

\newcommand{\sC}{\mathscr{C}}
\newcommand{\sE}{\mathscr{E}}
\newcommand{\sF}{\mathscr{F}}
\newcommand{\sG}{\mathscr{G}}
\newcommand{\sL}{\mathscr{L}}

\newcommand{\Ga}{\bG_{\mathrm{a}}}
\newcommand{\Gm}{\bG_{\mathrm{m}}}
\newcommand{\GGa}{\hat{\bG}_{\mathrm{a}}}
\newcommand{\GGm}{\hat{\bG}_{\mathrm{m}}}

% DOCUMENT OPTIONS %

\setlength{\parindent}{0em}

\title{Обобщение инвариантов формальных групп}
\author{}
\date{}

\begin{document}

\maketitle

\begin{section}{Обозначения и определения}

\begin{subsection}{Локальные поля и связанные с ними кольца}

\begin{denotation}\label{denote:1.0:XYZ}
    В качестве формальных переменных будут использоваться $x$, $t$, ${ X = (X_i) }$, ${ Y = (Y_i) }$, ${ Z = (Z_i) }$.
\end{denotation}

\begin{denotation}\label{denote:1.0:A}
    Пусть $A$ - коммутативное кольцо с единицей.
\end{denotation}

\begin{denotation}\label{denote:1.0:K}
    Пусть $K$ - локальное поле, т.е. полное дискретно нормированное поле с совершенным полем вычетов\footnote{
        \cite{FesenkoVostokov}, I.4.6.
    }.
\end{denotation}

Примеры:
\begin{enumerate}
    \item ${ K = \bQ_p }$.
    \item ${ K = \bQ_p(\zeta_{p^n}) }$, где $\zeta_{p^n}$ - примитивный корень из единицы степени $p^n$.
    \item ${ K = \bF_{p^n}((x)) }$.
\end{enumerate}

\begin{denotation}\label{denote:1.1:v_K}
    Пусть $v_K$ - нормализованное нормирование поля $K$, т.е. такое дискретное нормирование на $K$, что ${ v_K(K^\times) = \bZ }$\footnote{
        \cite{FesenkoVostokov}, I.3.1.
    }.
\end{denotation}

Примеры:
\begin{enumerate}
    \item Пусть ${ K = \bQ_p }$. Тогда ${ v_K = \ord_p }$. Более явно, для любого ${ \alpha \in \bQ_p^\times }$ имеет место ${ v_K(\alpha) = v }$, где ${ \alpha = p^v \varepsilon }$, ${ \varepsilon \in \bZ_p^\times }$.
    \item Пусть ${ K = \bQ_p(\zeta_{p^n}) }$. Тогда для любого ${ \alpha \in K }$ имеет место
    \begin{equation*}
        v_K(\alpha) =
        \ord_p\left(N_{K/\bQ_p}(\alpha)\right).
    \end{equation*}
    Здесь $N_{K/\bQ_p}$ - норма расширения ${ K / \bQ_p }$\footnote{
        \cite{Gouvea}, 6.3.
    }.
    \item Пусть ${ K = \bF_{p^n}((x)) }$. Тогда для любого ${ \alpha = \sum\limits_{i = v}^\infty a_i x^i \in K }$, имеет место ${ v_K(\alpha) = v }$.
\end{enumerate}

\begin{denotation}\label{denote:1.2:fO_K}
    Пусть $\fO_K$ - кольцо целых поля $K$, т.е.
    \begin{equation*}
        \fO_K =
        \{
            \alpha \in K
        |
            v_K(\alpha) \geq 0
        \}\footnote{
            \cite{FesenkoVostokov}, I.2.2.
        }.
    \end{equation*}
    Это локальное кольцо с максимальным идеалом
    \begin{equation*}
        \fM =
        \{
            \alpha \in K
        |
            v_K(\alpha) > 0
        \}.
    \end{equation*}
\end{denotation}

Примеры:
\begin{enumerate}
    \item Пусть ${ K = \bQ_p }$. Тогда ${ \fO_K = \bZ_p }$.
    \item Пусть ${ K = \bQ_p(\zeta_{p^n}) }$. Тогда ${ \fO_K = \bZ_p[\zeta_{p^n}] }$.
    \item Пусть ${ K = \bF_{p^n}((x)) }$. Тогда ${ \fO_K = \bF_{p^n}[[x]] }$.
\end{enumerate}

\begin{denotation}\label{denote:1.3:bar_K}
    Пусть $\overline{K}$ - поле вычетов для $K$, т.е. ${ \overline{K} = \fO_K / \fM }$\footnote{
        \cite{FesenkoVostokov}, I.2.2.
    }.
\end{denotation}

Примеры:
\begin{enumerate}
    \item Пусть ${ K = \bQ_p }$. Тогда ${ \overline{K} = \bF_p }$.
    \item Пусть ${ K = \bQ_p(\zeta_{p^n}) }$. Так как ${ K / \bQ_p }$ - вполне разветвлённое расширение\footnote{
        \cite{FesenkoVostokov}, IV.1.3.
    }, то ${ 1 = f(K / \bQ_p) = [\overline{K} : \bF_p] }$, поэтому ${ \overline{K} = \bF_p }$.
    \item Пусть ${ K = \bF_{p^n}((x)) }$. Тогда ${ \overline{K} = \bF_{p^n} }$.
\end{enumerate}

\begin{denotation}\label{denote:1.3:pi_K}
    Пусть $\pi$ - униформизующий элемент (локальный параметр) поля $K$, т.е. такой элемент поля $K$, для которого ${ \fM = \pi \fO_K }$.
\end{denotation}

Примеры:
\begin{enumerate}
    \item Пусть ${ K = \bQ_p }$. Тогда ${ \pi = p }$.
    \item Пусть ${ K = \bQ_p(\zeta_{p^n}) }$. Тогда ${ \pi = \zeta_{p^n} - 1 }$. В этом случае, ${ v_K(\pi) = 1 }$, и если представить ${ \alpha \in K^\times }$ в виде ${ \alpha = \pi^v \varepsilon }$, где ${ \varepsilon \in \fO_K^\times }$\footnote{
        \cite{FesenkoVostokov}, I.3.4.
    }, то ${ v_K(\alpha) = v }$.
    \item Пусть ${ K = \bF_{p^n}((x)) }$. Тогда ${ \pi = x }$.
\end{enumerate}

\begin{remark}\label{remark:1.before_e:K}
    С этого момента мы рассматриваем случай, когда ${ \fchar{K} = 0 }$, ${ \fchar{\overline{K}} = p > 0 }$, поэтому ${ K = \bF_{p^n}((x)) }$ в качестве примера уже не подходит.
\end{remark}

\begin{denotation}\label{denote:1.2:e}
    Пусть $e$ - абсолютный индекс ветвления поля $K$, т.е. ${ e = v_K(p) }$\footnote{
        \cite{FesenkoVostokov}, I.5.7.
    }.
\end{denotation}

Примеры:
\begin{enumerate}
    \item Пусть ${ K = \bQ_p }$. Тогда ${ e = 1 }$.
    \item Пусть ${ K = \bQ_p(\zeta_{p^n}) }$. Тогда
    \begin{equation*}
        e =
        \ord_p\left(N_{K/\bQ_p}(p)\right) =
        \ord_p\left(p^{[K : \bQ_p]}\right) =
        [K : \bQ_p] =
        (p - 1) p^{n - 1}.
    \end{equation*}
\end{enumerate}

\begin{denotation}\label{denote:1.4:N}
    Пусть ${ (N, \sigma) }$ - $\sigma$-подполе поля $K$, такое, что расширение ${ K/N }$ вполне разевтвлено. Это означает, что $N$ - локальное поле, ${ \fchar{N} = 0 }$, ${ \fchar{\overline{N}} = p }$, ${ f(K/N) = [\overline{K} : \overline{N}] = 1 }$, и существует ${ \sigma \in \End{N} }$ удовлетворяющее соотношению
    \begin{equation*}
        \sigma(a) - a^p \in \fM_N, \forall a \in \fO_N,
    \end{equation*}
    где $\fO_N$ - кольцо целых поля $N$, а $\fM_N$ - его максимальный идеал\footnote{
        \cite{BondarkoThesis}, 2.1.
    }. Поле ${ (N, \sigma) }$ всегда существует\footnote{
        \cite{BondarkoThesis}, 2.1.1.
    }, $N$ единственно, но $\sigma$, как правило, может быть выбрано более чем одним способом\footnote{
        \cite{FesenkoVostokov}, II.5.
    }. Мы будем использовать ${ \sigma(a) = a^p }$.
\end{denotation}

Примеры:
\begin{enumerate}
    \item Пусть ${ K = \bQ_p }$ или ${ K = \bQ_p(\zeta_{p^n}) }$. Тогда ${ N = \bQ_p }$.
\end{enumerate}

\begin{denotation}\label{denote:1.0:fA}
    Через $\fA$ будет обозначаться произвольное (не обязательно коммутативное) кольцо.
\end{denotation}

Примеры из рассматриваемых ниже случаев:
\begin{enumerate}
    \item ${ \fA = K, \fO_K }$.
    \item ${ \fA = K[[\Delta]], \fO_K[[\Delta]] }$.
\end{enumerate}

\begin{denotation}\label{denote:1.1:matrix_ring}
    Пусть $M_m(\fA)$ - кольцо матриц над $\fA$ размера ${ m \times m }$, а $M_{m \times n}(\fA)$ - модуль матриц размера ${ m \times n }$.
\end{denotation}

\begin{denotation}\label{denote:1.2:Delta}
    Пусть $\Delta$ - линейный оператор, возводящий формальные переменные в степень $p$, т.е.
    \begin{gather*}
        \Delta\left(\sum a_i x^i\right) =
        \sum a_i x^{pi},
        a_i \in \fA; \\
        \Delta\left(
            \sum a_{i_1 ... i_m} \prod X_j^{i_j}
        \right) =
        \sum a_{i_1 ... i_m} \prod X_j^{p i_j},
        a_{i_1 ... i_m} \in \fA.
    \end{gather*}
    Если ${ \Lambda = (\Lambda_{i j})_{i, j} = \left( \sum\limits_k a_{i j k} \Delta^k \right)_{i, j} \in M_m(\fA[[\Delta]]) }$, то под $\Lambda(X)$ мы будем подразумевать набор рядов
    \begin{equation*}
        \Lambda(X) =
        \lambda =
        (\lambda_i)_i =
        \left(
            \sum_{j, k} a_{i j k} X_j^{p^k}
        \right)_i \in
        M_{m \times 1}(\fA[[X]]) =
        (\fA[[X]])^m.
    \end{equation*}
    Если же ${ h = (h)_i = \left( \sum\limits_k a_{i k} \Delta^k \right)_i \in (\fA[[\Delta]])^m }$, то под $h(x)$ будет подразумеваться набор рядов
    \begin{equation*}
        h(x) =
        (h_i(x))_i =
        \left(
            \sum_k a_{i k} x^{p^k}
        \right)_i \in
        (\fA[[x]])^m.
    \end{equation*}
\end{denotation}

\begin{denotation}\label{denote:1.5:W}
    Обозначим через ${ N_\sigma[[\Delta]] }$ кольцо скрученных формальных степенных рядов Хонды\footnote{
        Honda’s twisted formal power series ring.
    }, т.е. такое некоммутативное кольцо степенных рядов, что ${ \Delta a = \sigma(a) \Delta }$ для всех ${ a \in N }$. Пусть ${ \fO_{N, \sigma}[[\Delta]] }$ - подкольцо ${ N_\sigma[[\Delta]] }$, состоящее из рядов с коэффициентами из $\fO_N$.
\end{denotation}

\begin{denotation}\label{denote:1.3:R}
    Положим
    \begin{equation*}
        R :=
        \fO_K[[\Delta]] \otimes_{\fO_K} K =
        \fO_K[[\Delta]] \otimes_{\bZ_p} \bQ_p =
        \bigcup_{i > 0} p^{-i} \fO_K[[\Delta]] \subset
        K[[\Delta]].
    \end{equation*}
\end{denotation}

\end{subsection}

\begin{subsection}{Формальные групповые законы}

\begin{definition}\label{def:2.0:formal_group}
    Набор из $m$ формальных степенных рядов от ${ 2 m }$ переменных ${ F \in (A[[X, Y]])^m }$ (${ X = (X_1, ..., X_m) }$, ${ Y = (Y_1, ..., Y_m) }$) называется \textit{формальным групповым законом} (или \textit{коммутативной формальной группой}) $F$ \textit{размерности $m$ над $A$}, если выполнено
    \begin{gather*}
        F(X, 0) = F(0, X) = X, \\
        F(F(X, Y), Z) = F(X, F(Y, Z)), \\
        F(X, Y) = F(Y, X).
    \end{gather*}
    Кроме того, для любого закона $F$ существует такой ряд ${ \iota_F(X) }$, что
    \begin{equation*}
        F(X, \iota_F(X)) = 0.
    \end{equation*}
\end{definition}

Примеры:
\begin{itemize}
    \item ${ m = 1 }$, ${ F = \GGa(x, y) = x + y }$.
    \item $m$ - любое, ${ F = \GGa^m = (\GGa(X_i, Y_i))_{i = \overline{1, m}} = (X_i + Y_i)_{i = \overline{1, m}} }$.
    \item ${ m = 1 }$, ${ F = \GGm(x, y) = x + y + x y = (1 + x)(1 + y) - 1 }$.
    \item $m$ - любое, ${ F = \GGm^m = (X_i + Y_i + X_i Y_i)_{i = \overline{1, m}} }$.
    \item Пусть $K$ - числовое локальное поле (т.е. ${ |\overline{K}| = q < \infty }$), ${ A = \fO_K }$, ${ m = 1 }$. Пусть ${ \gamma(x) = \sum\limits_{i > 0} a_i x^i \in A[[x]] }$, ${ a_1 \in A^\times }$. Положим
    \begin{equation*}
        f_\gamma(x) =
        \sum_{i \geq 0}
        \pi^{-i} \gamma(x^{q^i}).
    \end{equation*}
    Тогда
    \begin{equation*}
        F_\gamma(x, y) =
        f_\gamma^{-1}(f_\gamma(x) + f_\gamma(y))
    \end{equation*}
    - \textit{формальный групповой закон Любина-Тэйта}. Это единственный формальный групповой закон над $A$, для которого
    \begin{gather*}
        F_\gamma(x, y) \equiv
        x + y \mod \deg 2, \\
        e_\gamma(F_\gamma(x, y)) =
        F_\gamma(e_\gamma(x), e_\gamma(y)),
    \end{gather*}
    где ${ e_\gamma(x) = f_\gamma^{-1}(\pi f_\gamma(x)) \in A[[x]] }$\footnote{
        Подробности можно найти, например, в \cite{Hazewinkel}, 8.
    }.
    \item Пусть $K$ - числовое локальное поле, ${ q = |\overline{K}| }$, ${ A = \fO_K }$, $m$ - любое. Пусть ${ \Gamma \in M_m(\fM) }$ таково, что ${ \pi^{-1} \Gamma \in \GL{m}{A} }$. Положим
    \begin{equation*}
        f_\Gamma(X) =
        \sum\limits_{i \geq 0}
        \pi^{-i} (\pi^{-1} \Gamma)^{-i} X^{q^i},
    \end{equation*}
    где ${ X^{q^i} = \left( \begin{matrix} X_1^{q^i} \\ \vdots \\ X_m^{q^i} \end{matrix} \right) }$. Тогда
    \begin{equation*}
        F_\Gamma(X, Y) =
        f_\Gamma^{-1}(f_\Gamma(X) + f_\Gamma(Y))
    \end{equation*}
    - \textit{$m$-мерный формальный групповой закон Любина-Тэйта}. Это единственная формальный групповой закон над $A$, для которого
    \begin{equation*}
        e_\Gamma(F_\Gamma(X, Y)) =
        F_\Gamma(e_\Gamma(X), e_\Gamma(Y)),
    \end{equation*}
     где ${ e_\Gamma(X) = f_\Gamma^{-1}(\Gamma f_\Gamma(X)) }$\footnote{
        Подробности и дальнейшие обобщения можно найти, например, в \cite{Hazewinkel}, 13.
     }. У этого примера есть общий частный случай с предыдущим: если рассматривать ${ \gamma(x) = a x }$, ${ a \in A^\times }$, то можно получить те же самые групповые законы, что и при ${ m = 1 }$ в текущем примере.
\end{itemize}

\begin{denotation}\label{denote:2.1:curves_group}
    Пусть $F$ - $m$-мерный формальный групповой закон над $A$. Через ${ \sC(F) }$ обозначим группу, элементами которой являются так называемые \textit{кривые} - наборы из $m$ степенных рядов ${ \xi(x) = (\xi_i(x))_i \in A[[x]]^m }$, такие что ${\xi(0) = 0 }$. Сложение в этой группе производится по правилу
    \begin{equation*}
        \xi(x) +_F \eta(x) =
        F(\xi(x), \eta(x)).
    \end{equation*}
\end{denotation}

Примеры:
\begin{itemize}
    \item ${ \sC(\GGa^m) = \bigoplus\limits_{i = 1}^m A[[x]]^+ }$\footnote{
        ${ A[[x]]^+ = \{ \xi \in A[[x]] | \xi(0) = 0 \} }$.
    }.
\end{itemize}

\begin{denotation}\label{denote:2.2:[n]_F}
    Пусть $F$ - $m$-мерный формальный групповой закон над $A$, ${ \xi_\id = (x, ..., x) \in \sC(F) }$. Для всякого целого ${ n > 0 }$ положим
    \begin{equation*}
        [n]_F =
        \underbrace{
            \xi_\id +_F ... +_F \xi_\id
        }_{\text{$n$ слагаемых}} \in
        \sC(F).
    \end{equation*}
\end{denotation}

Примеры:
\begin{itemize}
    \item ${ [n]_{\GGa^m} = (n x, ..., n x) \in \sC(\GGa^m) }$.
    \item ${ [n]_{\GGm^m} = (S(x), ..., S(x)) \in \sC(\GGm^m) }$, где ${ S(x) = \sum\limits_{i = 1}^n \sigma_{n, i}(1, ..., 1) x^i }$, $\sigma_{n, i}$ - $i$-й элементарный симметрический многочлен от $n$ переменных.
\end{itemize}

\begin{definition}\label{def:2.2:Vfa}
    Пусть $F$ - $m$-мерный формальный групповой закон над $A$. Определим операторы $\oH{a}$, $\oV_n$, $\oF_n$, действующие на группе ${ \sC(F) }$, следующим образом:
    \begin{itemize}
        \item Для ${ \xi \in \sC(F) }$, ${ a \in A }$, пусть
        \begin{equation*}
            \oH{a} \xi(x) = \xi(a x).
        \end{equation*}
        \item Для ${ \xi \in \sC(F) }$, ${ n \in \bN }$, пусть
        \begin{equation*}
            \oV_n \xi(x) = \xi(x^n).
        \end{equation*}
        \item Для ${ \xi \in \sC(F) }$, ${ n \in \bN }$, пусть
        \begin{equation*}
            \xi(t_1 x^{\frac{1}{n}}) +_F
            ... +_F
            \xi(t_n x^{\frac{1}{n}}) =
            \alpha(
                \sigma_1, ..., \sigma_n,
                x^{\frac{1}{n}}
            ),
        \end{equation*}
        для некоторого ряда ${ \alpha \in A[t_1, ..., t_n][[x^{\frac{1}{n}}]]^m }$, где ${ t_1, ..., t_n }$ - формальные переменные, ${ \sigma_1, ..., \sigma_n }$ - элементарные симметрические многочлены от этих переменных. Тогда $\oF_n$ можно определить как
        \begin{equation*}
            \oF_n \xi(x) =
            \alpha(
                0, ..., 0, (-1)^{n - 1},
                x^{\frac{1}{n}}
            )\footnote{
                Подробности см. в \cite{Hazewinkel}, 15.1.
            }.
        \end{equation*}
    \end{itemize}
\end{definition}

Примеры:
\begin{itemize}
    \item ${ \oF_1 \xi = \xi }$.
    \item ${ \oF_n [k]_{\GGa^m} = 0 }$ при ${ n > 0 }$.
    \item Пусть ${ F = \GGm^m }$, тогда ${ \oF_n \xi_\id = (-1)^{n - 1} \xi_\id }$ при ${ n > 0 }$.
\end{itemize}

\begin{denotation}\label{denote:2.3:p-typical_curves_group}
    Пусть $F$ - $m$-мерный формальный групповой закон над $A$, $p$ - простое число. Положим
    \begin{equation*}
        \sC_p(F) =
        \left\{
            \xi \in \sC(F)
        \middle|
            \oF_q \xi(x) = 0
            \text{ для всех простых }
            q \neq p
        \right\}.
    \end{equation*}
    Элементы группы ${ \sC_p(F) }$ называются \textit{$p$-типическими кривыми}.
\end{denotation}

Примеры:
\begin{itemize}
    \item ${ [k]_{\GGa^m} \in \sC_p(\GGa^m) }$.
    \item ${ \xi_\id \not\in \sC_p(\GGm^m) }$.
\end{itemize}

\begin{definition}\label{def:2.1:formal_group_hom}
    Пусть $F$ - $m$-мерный формальный групповой закон над $A$, $G$ - $k$-мерный формальный групповой закон над $A$. Набор из $k$ формальных степенных рядов ${ f = (f_i(X))_i }$, где ${ f_i \in A[[X_1, ..., X_m]] }$, ${ f_i(X) \equiv 0 \mod \deg 1 }$, называется \textit{гомоморфизмом из $F$ в $G$}, если
    \begin{equation*}
        f(F(X, Y)) = G(f(X), f(Y)).
    \end{equation*}
    Гомоморфизм $f$ называется \textit{изоморфизмом}, если ${ m = k }$ и суещсвтует набор из $m$ рядов ${ g = (g_i(X))_i }$, такой что ${ (f \circ g)(X) = (g \circ f)(X) = X }$. Изоморфизм $f$ называется \textit{строгим}, если ${ f(X) \equiv I_m X \mod \deg 2 }$, где ${ I_m \in M_m(A) }$ - единичная матрица.
\end{definition}

Примеры:
\begin{itemize}
    \item Пусть ${ F = G }$. Тогда ${ f(X) = X }$ является изоморфизмом из $F$ в $G$.
    \item Пусть $m$ - любое, ${ F = \GGa^m }$, ${ k = 1 }$, ${ G = \GGa }$. Тогда
    \begin{equation*}
        f(X_1, ..., X_m) = X_1 + ... + X_m
    \end{equation*}
    является гомоморфизмом из $F$ в $G$.
    \item Пусть ${ m = 1 }$, ${ F = \GGa }$, $k$ - любое, ${ G = \GGa^k }$. Тогда
    \begin{equation*}
        f = (f_1(x), ..., f_k(x)) = (x, ..., x)
    \end{equation*}
    является гомоморфизмом из $F$ в $G$.
    \item Пусть $A$ является $\bQ$-алгеброй, ${ m = 1 }$, ${ F = \GGa }$, ${ k = 1 }$, ${ G = \GGm }$. Тогда ${ f(x) = \sum\limits_{i > 0} \dfrac{x^i}{i!} }$ является изоморфизмом из $F$ в $G$.
    \item Пусть $F_\gamma$ - групповой закон Любина-Тэйта над ${ A = \fO_K }$. Для всякого ${ a \in A }$ ряд
    \begin{gather*}
        [a]_\gamma(x) =
        f_\gamma^{-1}(a f_\gamma(x))
    \end{gather*}
    является эндоморфизмом формальной группового закона $F_\gamma$ для всякого ${ a \in A }$\footnote{
        \cite{Hazewinkel}, 8.1.5.
    }.
    \item Пусть $F_\Gamma$ - $m$-мерный групповой закон Любина-Тэйта над ${ A = \fO_K }$. Для всякой матрицы ${ \Beta \in M_m(A) }$, коммутирующей с $\Gamma$, ряд
    \begin{equation*}
        [\Beta]_\Gamma(X) =
        f_\Gamma^{-1}(\Beta f_\Gamma(X))
    \end{equation*}
    является эндоморфизмом формального группового закона ${ F_\Gamma }$\footnote{
        \cite{Hazewinkel}, 13.2.5.
    }.
\end{itemize}

\begin{definition}\label{def:2.1:base_change}
    Пусть ${ \varphi : A \rightarrow B }$ - гомоморфизм коммутативных колец с единицей,
    \begin{equation*}
        F(X, Y) =
        \sum\limits_{I, J} a_{I J} X^I Y^J
    \end{equation*}
    - формальный групповой закон над $A$ ($I$, $J$ - мультииндексы). Тогда
    \begin{equation*}
        \varphi_* F(X, Y) =
        \sum\limits_{I, J} \varphi(a_{I J}) X^I Y^J
    \end{equation*}
    является формальным групповым законом над $B$, полученным из ${ F(X, Y) }$ путём \textit{замены базы}.
\end{definition}

\begin{denotation}\label{denote:2.1:F}
    С этого момента $F$ будет $m$-мерным формальным групповым законом над $\fO_K$.
\end{denotation}

\begin{denotation}\label{denote:2.after_F:F_bar}
    Пусть $\overline{F}$ - редукция $F$ по модулю $\pi$, т.е. ${ \overline{F} = \kappa_* F }$, где ${ \kappa : \fO_K \rightarrow \overline{K} }$ - сюръекция ${ a \mapsto a \mod \pi }$.
\end{denotation}

\begin{denotation}\label{denote:2.4:ht(F)}
    Обозначим через ${ h = h(F) }$ высоту группового закона $F$, то есть, величину
    \begin{equation*}
        h =
        \dim_{\overline{K}} \left(
            \sC_p(\overline{F}) /
            [p]_{\overline{F}} \sC_p(\overline{F})
        \right),
    \end{equation*}
    если она конечна\footnote{
        Это определение взято из \cite{Hazewinkel}, 28.2.9.
    }.
\end{denotation}

Примеры:
\begin{itemize}
    \item ${ h(\GGa^m) = \infty }$.
    \item ${ h(\GGm^m) = 1 }$.
\end{itemize}

\begin{denotation}\label{def:2.2:lambda_F}
    Пусть $\lambda_F$ - логарифм $F$, т.е. $\lambda_F(X)$ - строгий изоморфизм из ${ \iota_* F }$ в $\GGa^m$ (который существует и единственен\footnote{
        \cite{Hazewinkel}, 11.1.6.
    }), где ${ \iota : \fO_K \rightarrow \fO_K \otimes_{\bZ} \bQ = K }$ - вложение.
\end{denotation}

Примеры:
\begin{itemize}
    \item Пусть ${ F = \GGa^m }$. Тогда ${ \lambda_F(X) = X }$.
    \item Пусть ${ F = \GGm^m }$. Тогда ${ \lambda_F(X) = (\lambda_i(X))_i }$, где ${ \lambda_i(X) = \sum\limits_{j > 0} (-1)^{j + 1} \dfrac{X_i^j}{j} }$.
    \item Пусть ${ F = F_\Gamma }$. Тогда ${ \lambda_F(X) = f_\Gamma(X) }$.
\end{itemize}

\begin{definition}\label{def:2.3:curvilinear}
    Формальный групповой закон $F$ называется \textit{криволинейным}, если $\lambda_F$ имеет вид
    \begin{equation*}
        \lambda_F(X) =
        \sum_{i > 0}
        A_i X^i,
    \end{equation*}
    где ${ A_i \in M_m(K) }$, и ${ X^i = \left( \begin{matrix} X_1^i \\ \vdots \\ X_m^i \end{matrix} \right) }$\footnote{
        \cite{Hazewinkel}, 12.1.2.
    }.
\end{definition}

\begin{definition}\label{def:2.4:p-typical}
    Пусть ${ \ell \in \bZ }$ - некоторое простое число (не обязательно совпадающее с ${ p = \fchar{\overline{K}} }$). Формальный групповой закон $F$ называется \textit{$\ell$-типическим}, если $\lambda_F$ имеет вид
    \begin{equation*}
        \lambda_F(X) =
        \sum_{i \geq 0} A_i X^{\ell^i},
    \end{equation*}
    где ${ A_i \in M_m(K) }$, и ${ X^{\ell^i} = \left( \begin{matrix} X_1^{\ell^i} \\ \vdots \\ X_m^{\ell^i} \end{matrix} \right) }$\footnote{
        \cite{Hazewinkel}, 15.2.6.
    }.
\end{definition}

\begin{remark}\label{remark:2.after_p-typical:F_is_p-typical}
    Любой формальный групповой закон $F$ над $\fO_K$ строго изоморфен некоторому $p$-типическому\footnote{
        \cite{Hazewinkel}, 16.4.14.
    }. Поэтому мы будем изначально рассматривать только $p$-типические формальные групповые законы (такие как $\GGa^m$ и $F_\Gamma$). Согласно \ref{denote:1.2:Delta} и \ref{def:2.4:p-typical}, это означает, что ${ \lambda_F = \Lambda_F(X) }$ для некоторой матрицы ${ \Lambda_F \in M_m(K[[\Delta]]) }$.
\end{remark}

\begin{definition}\label{def:2.5:rank_of_hom}
    Пусть $G$ - $k$-мерный формальный групповой закон над $\fO_K$, ${ \alpha : F \rightarrow G }$ - гомоморфизм. Тогда существуют изоморфизмы ${ f : F \rightarrow F' }$ и ${ g : G \rightarrow G' }$, такие, что $F'$ и $G'$ криволинейные формальные групповые законы, а ${ \alpha' = g \circ \alpha \circ f^{-1} : F' \rightarrow G' }$ имеет вид
    \begin{gather*}
        \alpha'_1(X) = X_1^{p^{n_1}}, \\
        ..., \\
        \alpha'_r(X) = X_r^{p^{n_r}}, \\
        \alpha'_s(X) = 0, r < s \leq k,
    \end{gather*}
    где ${ n_1 \leq n_2 \leq ... \leq n_r }$ - некоторые натуральные числа\footnote{
        \cite{Hazewinkel}, 28.2.6.
    }. Мы будем называть число ${ \rank{\alpha} = r }$ \textit{рангом} гомоморфизма $\alpha$.
\end{definition}

\begin{definition}\label{def:2.6:isogeny}
    Пусть $G$ - $k$-мерный формальный групповой закон над $\fO_K$, ${ \alpha : F \rightarrow G }$ - гомоморфизм. $\alpha$ называется \textit{изогенией}, если ${ m = k = \rank{\alpha} }$.
\end{definition}

\end{subsection}

\begin{subsection}{Формальные группы}

\begin{denotation}\label{denote:3.0:Nil_A}
    Обозначим через $\Nil{A}$ категорию нильпотентных коммутативных $A$-алгебр\footnote{
        $A$-алгеброй называется кольцо $B$, снабженное некоторым гомоморфизмом колец ${ f : A \rightarrow B }$ (\cite{Lang}, глава V, \S 1), называемым \textit{структурным морфизмом}. $A$-алгебра $B$ нильпотентна, если существует такое ${ n > 0 }$, что ${ b_1 \cdot ... \cdot b_n = 0 }$ для любых ${ b_1, ..., b_n \in B }$.
    }.
\end{denotation}

\begin{denotation}\label{denote:3.1:Mod_A}
    Обозначим через $\Mod{A}$ категорию $A$-модулей.

    Так как любой $A$-модуль ${ M \in \Mod{A} }$ можно рассматривать как нильпотентную $A$-алгебру, полагая ${ M^2 = 0 }$, то $\Mod{A}$ можно рассматривать как подкатегорию в $\Nil{A}$.
\end{denotation}

\begin{definition}\label{def:3.1:augmented_algebra}
    $A$-алгебра $B$ со структурным морфизмом ${ f : A \rightarrow B }$ называется \textit{аугментированной}, если задан гомоморфизм $A$-алгебр ${ \varepsilon : B \rightarrow A }$, называемый \textit{аугментацией}, такой, что ${ \varepsilon \circ f = \id_A }$. Мы также будем пользоваться обозначением ${ B^+ = \Ker{\varepsilon} }$.

    Аугментированная $A$-алгебра $B$ называется \textit{нильпотентной}, если ${ B^+ \in \Nil{A} }$.
\end{definition}

\begin{denotation}\label{denote:3.after_augmented_algebra:Compl_A}
    Обозначим через $\Compl{A}$ категорию полных аугментированных $A$-алгебр. То есть, объектами $\Compl{A}$ являются пары ${ (C, \{ \fc_i \})_{i \in \bN} }$, где $C$ - аугментированная $A$-алгебра; ${ \fc_1 = C^+ }$; ${ \fc_1 \supset \fc_2 \supset \fc_3 \supset ... }$ - убывающая последовательность идеалов в $C$, причём ${ \fc_1 / \fc_i \in \Nil{A} }$ для любого ${ i \in \bN }$, и ${ C = \varprojlim C / \fc_i }$.

    Полагая ${ C = A \oplus N }$, ${ \fc_1 = N }$ и ${ \fc_i = 0 }$ при ${ i > 1 }$, мы получаем объект из $\Compl{A}$ для любого ${ N \in \Nil{A} }$. Поэтому можно рассматривать $\Nil{A}$ как подкатегорию в $\Compl{A}$.
\end{denotation}

\begin{denotation}\label{denote:3.0:Sets}
    Обозначим через $\Sets$ категорию множеств.
\end{denotation}

\begin{denotation}\label{denote:3.0:Ab}
    Обозначим через $\Ab$ категорию абелевых групп.
\end{denotation}

\begin{definition}\label{def:3.1:representable_functor}
    Пусть $C$ - аугментированная нильпотентная $A$-алгебра. Она определяет функтор
    \begin{equation*}
        \Spf{C} : \Nil{A} \rightarrow \Sets
    \end{equation*}
    следующим образом:
    \begin{gather*}
        \Spf{C} : N \mapsto \Hom{C^+}{N}, \\
        \Spf{C} :
        (f : N \rightarrow M) \mapsto
        (g \mapsto f \circ g).
    \end{gather*}
    Функтор ${ \sF : \Nil{A} \rightarrow \Sets }$ называется \textit{представимым}, если он изоморфен функтору вида $\Spf{C}$ для некоторой аугментированной нильпотентной $A$-алгебры $C$.

    Если же ${ (C, \{ \fc_i \}) \in \Compl{A} }$, то такая алгебра определяет функтор
    \begin{equation*}
        \Spf{C, \{\fc_i\}} : \Nil{A} \rightarrow \Sets
    \end{equation*}
    как предел представимых функторов:
    \begin{equation*}
        \Spf{C, \{\fc_i\}} :
        N \mapsto
        \varinjlim \left(
            \Spf{C / \fc_i}(N)
        \right).
    \end{equation*}
    Функтор ${ \sF : \Nil{A} \rightarrow \Sets }$ называется \textit{пропредставимым}, если он изоморфен функтору вида ${ \Spf{C, \{\fc_i\}} }$ для некоторой полной аугментированной $A$-алгебры ${ (C, \{ \fc_i \}) \in \Compl{A} }$. Очевидно, любой представимый функтор является также пропредставимым\footnote{
        Если $C$ - аугментированная нильпотентная $A$-алгебра, то ${ \Spf{C} = \Spf{C, \{\fc_i\}} }$, где ${ \fc_1 = C^+ }$ и ${ \fc_i = 0 }$ при ${ i > 1 }$.
    }.

    Пропредставимый функтор ${ \sF \cong \Spf{C, \{\fc_i\}} }$ называется \textit{строго пропредставимым}, если ${ \fc_1 / \fc_i }$ является конечнопорождённым проективным $A$-модулем для каждого ${ i \geq 1 }$\footnote{
        О конечнопорождённых и проективных модулях - \cite{Lang}, глава III.
    }.

    Соответственно, представимый функтор ${ \sF \cong \Spf{C} }$ называется \textit{строго представимым}, если он является строго пропредставимым, то есть, если $C^+$ - конечнопорождённый проективный $A$-модуль.
\end{definition}

\begin{definition}\label{def:3.1:formal_group}
    \textit{Формальной группой} над $A$ называется точный функтор ${ \sF : \Nil{A} \rightarrow \Ab }$, коммутирующий с бесконечными прямыми суммами\footnote{
        Разъяснения см. в \cite{Zink}, 2.2.
    }.
\end{definition}

Примеры:
\begin{itemize}
    \item ${ \Ga^m : N \mapsto (N, +_N)^m }$, где $+_N$ - операция сложения в алгебре $N$.
    \item ${ \Gm^m : N \mapsto (1 + N, \cdot)^m }$, где ${ 1 + N }$ - множество формальных сумм вида ${ 1 + u }$, ${ u \in N }$, и умножение выполняется по правилу
    \begin{equation*}
        (1 + u) \cdot (1 + v) =
        1 + (u +_N v +_N u \cdot_N v),
    \end{equation*}
    где ${ u, v \in N }$, $+_N$, $\cdot_N$ - соответствующие операции сложения и умножения в алгебре $N$.
    \item Пусть $K$ - числовое локальное поле, ${ q = |\overline{K}| }$, ${ A = \fO_K }$, ${ m \in \bN }$. Зафиксируем некоторое число ${ h \in \bN }$, и некоторую матрицу ${ \Gamma \in M_m(\fM) }$, для которой ${ \pi^{-1} \Gamma \in \GL{m}{A} }$. Пусть ${ e_{\Gamma, h} \in (K[[X_1, ..., X_m]])^m }$ - такой набор рядов, для которого
    \begin{gather*}
        e_{\Gamma, h}(X) \equiv \Gamma X \mod \deg 2, \\
        e_{\Gamma, h}(X) \equiv X^{q^h} \mod \pi.
    \end{gather*}
    Тогда для каждого ${ N \in \Nil{A} }$ мы можем построить отображение ${ \varepsilon_{\Gamma, h, N} : N^m \rightarrow N^m }$ по правилу
    \begin{equation*}
        \varepsilon_{\Gamma, h, N}(a) =
        e_{\Gamma, h}(a).
    \end{equation*}
    $\varepsilon_{\Gamma, h, N}$ определено корректно, так как $N$ - нильпотентная $A$-алгебра. Согласно теореме 13.3.3 из \cite{Hazewinkel}, существует единственная формальная группа $\sF_{\Gamma, h}$, такая, что ${ \varepsilon_{\Gamma, h, N} : \sF_{\Gamma, h}(N) \rightarrow \sF_{\Gamma, h}(N) }$ является отображением абелевых групп для каждого ${ N \in \Nil{A} }$\footnote{
        Отображение $\varepsilon_{\Gamma, h, N}$ переопределено корректно, поскольку $\sF_{\Gamma, h}(N)$ и $N^m$ совпадают как множества - это тоже следует из теоремы.
    }. $\sF_{\Gamma, h}$ можно называть \textit{$m$-мерной формальной группой Любина-Тэйта}.
    \item Для всякого ${ N \in \Nil{A} }$ обозначим через $\Lambda(N)$ подгруппу по умножению в ${ N[t] }$, состоящую из многочленов вида ${ 1 + u_1 t + ... + u_n t^n }$.
\end{itemize}

\begin{definition}\label{def:3.after_formal_group:representable_formal_group}
    Формальная группа $\sF$ над $A$ называется представимой (пропредставимой / строго представимой / строго пропредставимой), если композиция $\sF$ с забывающим функтором является представимым (пропредставимым / строго представимым / строго пропредставимым) функтором.
\end{definition}

Примеры:
\begin{itemize}
    \item Формальная группа $\Ga^m$ над произвольным $A$ пропредставима.
    \item Формальная группа $\Gm^m$ над произвольным $A$ пропредставима.
    \item Формальная группа $\sF_{\Gamma, h}$ над $\fO_K$\footnote{
        Здесь вновь $K$ - числовое локальное поле.
    } строго пропредставима.
\end{itemize}

\begin{remark}\label{remark:3.2:formal_group_extension}
   Пусть $\sF$ - формальная группа над $A$. Можно считать, что функтор $\sF$ действует на $\Compl{A}$, полагая
   \begin{equation*}
       \sF((C, \{ \fc_i \})) =
       \varprojlim \sF(\fc_1 / \fc_i)
   \end{equation*}
   для всякого ${ (C, \{ \fc_i \})_{i \in \bN} }$.
\end{remark}

\begin{definition}\label{def:3.2:tangent_functor}
    Пусть $\sF$ - формальная группа над $A$. Её \textit{касательным функтором $t_\sF$} называется ограничение $\sF$ на подкатегорию $\Mod{A}$.
\end{definition}

\begin{definition}\label{def:3.3:formal_group_of_finite_dimension}
    Говорят, что формальная группа $\sF$ над $A$ \textit{конечномерна размерности $m$}, если $t_\sF(A)$ - свободный конечно порождённый проективный $A$-модуль ранга $m$\footnote{
        См. \cite{Lang}, глава III.
    }.
\end{definition}

\begin{denotation}\label{denote:3.before_lie_algebra_functor:forgetful_functor}
    Пусть ${ \forget{A} : \Nil{A} \rightarrow \Mod{A} }$ - функтор, забывающий умножение, и превращающий любую $A$-алгебру ${ N \in \Nil{A} }$ в $A$-модуль. Это же обозначение мы сохраним для аналогичного функтора, действующего из $\Compl{A}$ в $\Mod{A}$.
\end{denotation}

\begin{definition}\label{denote:3.2:lie_algebra_functor}
    Пусть $\sF$ - формальная группа над $A$. \textit{Алгеброй Ли группы $\sF$} называется функтор ${ \Lie{\sF} = t_\sF \circ \forget{A} }$.
\end{definition}

\begin{claim}\label{claim:3.4:groups_and_laws_correspondence}
    Формальные группы связаны с формальными групповыми законами следующим образом:
    \begin{itemize}
        \item Пусть $\sF$ - конечномерная формальная группа над $A$ размерности $m$. Пусть ${ \fa = A[[X, Y]]^+ }$. Тогда ${ \fa / \fa^n \in \Nil{A} }$ для любого ${ n \in \bN }$. Обозначим через $+_{\sF, n}$ операцию сложения в группе ${ \sF(\fa / \fa^n) \in \Ab }$\footnote{
            Так как $\sF$ имеет размерность $m$, то ${ \sF(\fa / \fa^n) = \left( \fa / \fa^n \right)^m }$ как множества, см. \cite{Zink}, 2.32.
        }, и положим
        \begin{equation*}
            \left(
                F_1^{(n)}, ..., F_m^{(n)}
            \right) =
            (X_1, ..., X_m) +_{\sF, n} (Y_1, ..., Y_m).
        \end{equation*}
        Если теперь для каждого $i$ построить единственный ряд ${ F_i \in A[[X, Y]] }$, такой, что ${ F_i \equiv F_i^{(n)} \mod \fa^n }$, то мы получим формальный групповой закон ${ F = (F_i)_i }$ размерности $m$ над $A$, соответствующий формальной группе $\sF$.
        \item Пусть ${ F = (F_i)_i }$ - формальный групповой закон размерности $m$ над $A$, ${ N \in \Nil{A} }$. Так как $N$ нильпотентная $A$-алгебра, то выражению ${ F_i(a, b) }$ можно придать очевидный смысл для любых ${ a, b \in N^m }$, ${ i = 1,..., m }$, и, таким образом, мы получаем некоторый элемент из $N$. Полагая
        \begin{equation*}
            a +_{F, N} b =
            (F_1(a, b), ..., F_m(a, b)),
        \end{equation*}
        для любых ${ a, b \in N^m }$, мы определяем таким образом сложение $+_{F, N}$ на $N^m$. ${ (N^m, +_{F, N}) }$ является абелевой группой, следовательно, мы получили некоторый функтор ${ \sF : \Nil{A} \rightarrow \Ab }$. На морфизмах этот функтор действует следующим образом: если ${ f : N \rightarrow M }$ - гомоморфизм нильпотентных $A$-алгебр, то
        \begin{equation*}
            \sF(f) :
            (a_1, ..., a_m) \mapsto
            (f(a_1), ..., f(a_m))
        \end{equation*}
        для любого ${ (a_1, ..., a_m) \in \sF(N) }$. Функтор $\sF$ является формальной группой над $A$ размерности $m$, соответствующей формальному груповому закону $F$.
    \end{itemize}
\end{claim}

Примеры:
\begin{itemize}
    \item Формальному групповому закону $\GGa^m$ соответствует формальная группа $\Ga^m$.
    \item Формальному групповому закону $\GGm^m$ соответствует формальная группа $\Gm^m$.
    \item Формальному групповому закону Любина-Тэйта $F_\Gamma$ соответствует формальная группа Любина-Тэйта ${ \sF_{\Gamma, 1} }$.
\end{itemize}

\begin{denotation}\label{denote:3.3:curves_group}
    Пусть $\sF$ - формальная группа над $A$. Через $M_\sF$ обозначим группу $\sF(A[[x]]^+)$\footnote{
        ${ A[[x]] \in \Compl{A} }$. Аугментация ${ \varepsilon : A[[x]] \rightarrow A }$ имеет вид
        \begin{equation*}
            \varepsilon :
            \sum_{i \geq 0} a_i x^i \mapsto
            a_0,
        \end{equation*}
        и ${ \fc_i = x^i A[[x]] }$ для всех ${ i \in \bN }$.
    } и будем называть её элементы \textit{кривыми}.
\end{denotation}

Примеры:
\begin{itemize}
    \item ${ M_{\Ga^m} = \sC(\GGa^m) = \bigoplus\limits_{i = 1}^m A[[x]]^+ }$.
\end{itemize}

\begin{definition}\label{def:3.4:formal_group_hom}
    Пусть $\sF$, $\sG$ - формальные группы над $A$. Естественное преобразование ${ \varphi : \sF \rightarrow \sG }$ называется \textit{гомоморфизмом} формальных групп. Таким образом, для каждого ${ N \in \Nil{A} }$ задан гомоморфизм групп ${ \varphi_N : \sF(N) \rightarrow \sG(N) }$, причём для любого гомофоризма алгебр ${ f : N \rightarrow M }$, ${ N, M \in \Nil{A} }$, следующая диаграмма коммутативна:
    \begin{center}
        \begin{tikzcd}
            \sF(N)
                \arrow{r}{\varphi_N}
                \arrow[swap]{d}{\sF(f)} &
            \sG(N)
                \arrow{d}{\sG(f)} \\
            \sF(M)
                \arrow{r}{\varphi_M} &
            \sG(M)
        \end{tikzcd}
    \end{center}
    Естественный изоморфизм ${ \varphi : \sF \rightarrow \sG }$ называется \textit{изоморфизмом} формальных групп. В этом случае, для каждого ${ N \in \Nil{A} }$ гомоморфизм $\varphi_N$ является изоморфизмом абелевых групп.
\end{definition}

Примеры:
\begin{itemize}
    \item Пусть $\sF$ - произвольная формальная группа над $A$, тогда ${ \id_\sF : \sF \rightarrow \sF }$ - соответствующий ей тождественный автоморфизм, который для каждого ${ N \in \Nil{A} }$ имеет вид ${ (\id_\sF)_N = \id_{\sF(N)} }$.
    \item Пусть $A$ является $\bQ$-алгеброй, ${ m \in \bN }$, ${ \sF = \Ga^m }$, ${ \sG = \Gm^m }$. Для каждого ${ N \in \Nil{A} }$ определим гомоморфизм ${ \varphi_N : \sF(N) \rightarrow \sG(N) }$ следующим образом:
    \begin{equation*}
        \varphi_N :
        (a_1, ..., a_m) \mapsto
        \left(
            \sum_{i > 0} \frac{a_1^i}{i!},
            ...,
            \sum_{i > 0} \frac{a_m^i}{i!}
        \right).
    \end{equation*}
    Определение корректно, так как $N$ - нильпотентная $A$-алгебра. В таком случае, $\varphi$ является строгим изоморфизмом из $\sF$ в $\sG$.
    \item Пусть $\sF_{\Gamma, 1}$ - $m$-мерная формальная группа Любина-Тэйта над ${ A = \fO_K }$. Для всякой матрицы ${ B \in M_m(A) }$, коммутирующей с $\Gamma$, можно построить эндоморфизм ${ [\beta]_\Gamma \in \End{\sF_{\Gamma, 1}} }$ следующим образом. Пусть ${ N \in \Nil{A} }$, тогда гомоморфизм $ \left( [\beta]_\Gamma \right)_N : \sF_{\Gamma, 1}(N) \rightarrow \sF_{\Gamma, 1}(N) $ должен действовать по правилу
    \begin{equation*}
        \left( [\beta]_\Gamma \right)_N :
        a \mapsto
        [B]_\Gamma(a).
    \end{equation*}
\end{itemize}

\begin{claim}\label{claim:3.5:group_and_law_homs_correspondence}
    Гомоморфизмы формальных групп связаны с гомоморфизмами формальных групповых законов следующим образом:
    \begin{itemize}
        \item Пусть $\sF$ - формальная группа над $A$ размерности $m$, $\sG$ - формальная группа над $A$ размерности $k$, $F$, $G$ - соответствующие им формальные групповые законы над $A$\footnote{
            Построенные таким же образом, как в утверждении \ref{claim:3.4:groups_and_laws_correspondence}.
        }, ${ \varphi : \sF \rightarrow \sG }$ - гомоморфизм. Пусть ${ \fb = A[[X]]^+ }$. Тогда ${ \fb / \fb^n \in \Nil{A} }$ для любого ${ n \in \bN }$. Положим
        \begin{equation*}
            \left(
                f_1^{(n)}, ..., f_k^{(n)}
            \right) =
            \varphi_{\fb / \fb^n} (X_1, ..., X_m).
        \end{equation*}
        Если теперь для каждого $i$ построить единственный ряд ${ f_i \in A[[X]] }$, такой, что ${ f_i \equiv f_i^{(n)} \mod \fb^n }$, то мы получим гомоморфизм ${ f = (f_i)_i }$ из $F$ в $G$\footnote{
            То, что набор рядов ${ (f_1, ..., f_k) }$ является гомоморфизмом формальных групповых законов следует из того, что $\varphi_{\fb / \fb^n}$ - гомоморфизм групп.
        }.
        \item Пусть $F$ - формальный групповой закон размерности $m$ над $A$, $G$ - формальный групповой закон размерности $k$ над $A$, $\sF$, $\sG$ - соответствующие им формальные группы над $A$\footnote{
            Тоже построенные как в утверждении \ref{claim:3.4:groups_and_laws_correspondence}.
        }, ${ f = (f_i)_i }$ - гомоморфизм из $F$ в $G$, ${ N \in \Nil{A} }$. Вновь мы можем для любых ${ a \in N^m }$, ${ i = 1, ..., k }$, придать смысл выражению ${ f_i(a) }$, так как алгебра $N$ нильпотентна, т.е. ${ f(a) \in N^k }$. Таким образом, полагая
        \begin{equation*}
            \varphi_N(a) = f(a)
        \end{equation*}
        для любого ${ a \in \sF(N) }$, мы определяем гомоморфизм ${ \varphi_N : \sF(N) \rightarrow \sG(N) }$ для произвольного ${ N \in \Nil{A} }$. Прямой проверкой можно убедиться, что набор таких гомоморфизмов определяет естественное преобразование ${ \varphi : \sF \rightarrow \sG }$.
    \end{itemize}
\end{claim}

\begin{claim}
    Пусть ${ \varphi = \{ \varphi_N \}_{N \in \Nil{A}} : \sF \rightarrow \sG }$ - гомоморфизм формальных групп над $A$. Для любого ${ N \in \Nil{A} }$ пусть ${ N^0 }$ - кольцо, состоящее из тех же элементов, что и $N$, с той же операцией сложения, но с умножением, задаваемым по правилу ${ N^0 \cdot N^0 = 0 }$. Тогда ${ \Lie{\varphi} = \{ \varphi_{N^0} \}_{N \in \Nil{A}} : \Lie{\sF} \rightarrow \Lie{\sG} }$ также является гомоморфизмом формальных групп над $A$.
\end{claim}
\begin{proof}
    Тривиально.
\end{proof}

\begin{definition}\label{def:3.6:isogeny}
    Гомоморфизм ${ \varphi : \sF \rightarrow \sG }$ формальных групп над $A$ одинаковой конечной размерности называется \textit{изогенией}, если функтор $\Ker{\varphi}$\footnote{
        Функтор ${ \Ker{\varphi} : \Nil{A} \rightarrow \Sets }$ определяется следующим образом:
        \begin{gather*}
            \Ker{\varphi} : N \mapsto \Ker{\varphi_N}, \\
            \Ker{\varphi} :
            (f : N \rightarrow M) \mapsto
            \sF(f)|_{\Ker{\varphi_N}}.
        \end{gather*}
    } представим.
\end{definition}

\begin{theorem}\label{theorem:3.5:first_main_theorem_of_cartier_theory}
    Пусть $\sF$ - формальная группа над $A$. Тогда отображение ${ \iota_\sF : \Hom{\Lambda}{\sF} \rightarrow \sF(A[[x]]^+) }$ вида
    \begin{equation*}
        \iota_\sF :
        \Phi \mapsto
        \Phi_{A[[x]]^+}(1 - x t)
    \end{equation*}
    является изоморфизмом абелевых групп.
\end{theorem}
\begin{proof}
    Первая основная теорема теории Картье, см. \cite{Zink}, теорема 3.5.
\end{proof}

\begin{denotation}\label{denote:3.6:cartier_ring}
    Обозначим через ${ \Cart(A) = (\End{\Lambda})^\mathrm{op} }$\footnote{
        Очевидно, $\Lambda$ рассматривается над $A$, и $R^\mathrm{op}$ для кольца $R$ означает кольцо, совпадающее с $R$ как группа по сложению, но умножение в котором выполняется в обратном порядке относительно умножения в $R$.
    } \textit{кольцо Картье}, соответствующее $A$. Воспользуемся теоремой \ref{theorem:3.5:first_main_theorem_of_cartier_theory} для обозначения некоторых особых элементов этого кольца:
    \begin{itemize}
        \item Для ${ a \in A }$, пусть
        \begin{equation*}
            \eH{a} = \iota_\Lambda^{-1}(1 - a x t).
        \end{equation*}
        \item Для ${ n \in \bN }$, пусть
        \begin{equation*}
            \eV_n = \iota_\Lambda^{-1}(1 - x^n t).
        \end{equation*}
        \item Для ${ n \in \bN }$, пусть
        \begin{equation*}
            \eF_n = \iota_\Lambda^{-1}(1 - x t^n).
        \end{equation*}
    \end{itemize}
    Имеет место следующее представление:
    \begin{equation*}
        \Cart(A) =
        \left\{
            \sum_{r = 1}^\infty
            \sum_{s = 1}^{N_r}
            \eV_r \eH{a_{r, s}} \eF_s
        \middle|
            N_r \in \bN,
            a_{r, s} \in A
        \right\}\footnote{
            Теорема 3.12 в \cite{Zink}.
        }.
    \end{equation*}
\end{denotation}

\begin{remark}\label{remark:3.7:action_of_cartier_ring_on_curves}
    Пусть $\sF$ - формальная группа над $A$. Тогда теорема \ref{theorem:3.5:first_main_theorem_of_cartier_theory} позволяет рассматривать группу кривых $M_\sF$ как модуль над кольцом Картье $\Cart(A)$\footnote{
        Также это позволяет смотреть на элементы кольца $\Cart(A)$ как на операторы, действующие на кривых формальной группы $\sF$.
    }. А именно, для ${ \Phi \in \Cart(A) }$ и ${ \xi \in M_\sF }$ мы определяем их произведение как
    \begin{equation*}
        \Phi \xi =
        \iota_\sF \left(
            \iota_\sF^{-1}(\xi) \circ \Phi
        \right) \in M_\sF.
    \end{equation*}
\end{remark}

\begin{denotation}\label{denote:3.8:p-typical_curves_group}
    Пусть $\sF$ - формальная группа над $A$, $p$ - простое число. Положим
    \begin{equation*}
        M_{\sF, p} = \left\{
            \xi \in M_\sF
        \middle|
            \text{${ \eF_n \xi = 0 }$ для всех ${ (n, p) = 1 }$, ${ n > 1 }$}
        \right\}.
    \end{equation*}
    Элементы группы $M_{\sF, p}$ называются \textit{$p$-типическими кривыми}.
\end{denotation}

\begin{definition}\label{def:3.3:base_change}
    Пусть ${ \varphi : A \rightarrow B }$ - гомоморфизм коммутативных колец с единицей. Пусть ${ N \in \Nil{B} }$, что влечёт существование гомоморфизма колец ${ f : B \rightarrow N }$. Тогда гомоморфизмом также является отображение ${ f \circ \varphi }$, следовательно, ${ N \in \Nil{A} }$. Таким образом, мы определили функтор
    \begin{equation*}
        b_\varphi : \Nil{B} \rightarrow \Nil{A}.
    \end{equation*}
    Пусть $\sF$ - формальная группа над $A$, тогда ${ \varphi_* \sF = \sF \circ b_\varphi }$ является формальной группой над $B$, полученной из $\sF$ путём \textit{замены базы}.
\end{definition}

\begin{claim}
    Пусть $\sF$ - формальная группа над $A$. Тогда:
    \begin{itemize}
        \item если ${ \varphi : A \rightarrow B }$ и ${ \psi : B \rightarrow C }$ - гомоморфизмы колец, то
        \begin{equation*}
            \psi_* \left( \varphi_* \sF \right) =
            (\psi \circ \varphi)_* \sF;
        \end{equation*}
        \item если ${ \varphi : A \rightarrow B }$ - гомоморфизм колец, то
        \begin{equation*}
            \varphi_* \Lie{\sF} =
            \Lie{\varphi_* \sF};
        \end{equation*}
        \item если $\sG$ - другая формальная группа над $A$, ${ \alpha = \{ \alpha_N \}_{N \in \Nil{A}} : \sF \rightarrow \sG }$ - гомомофризм формальных групп над $A$, ${ \varphi : A \rightarrow B }$ - гомоморфизм колец, то ${ \varphi_* \alpha = \{ \alpha_{b_\varphi(N)} \}_{N \in \Nil{B}} : \varphi_* \sF \rightarrow \varphi_* \sG }$ является гомоморфизмом формальных групп над $B$.
    \end{itemize}
\end{claim}
\begin{proof}
    Проверяется легко.
\end{proof}

\begin{denotation}\label{denote:3.2:sF}
    С этого момента $\sF$ будет конечномерной формальной группой размерности $m$ над $\fO_K$.
\end{denotation}

\begin{denotation}\label{denote:3.after_sF}
    Пусть $\overline{\sF}$ - редукция $\sF$ по модулю $\pi$, т.е. ${ \overline{\sF} = \kappa_* \sF }$, где ${ \kappa : \fO_K \rightarrow \overline{K} }$ - сюръекция ${ a \mapsto a \mod \pi }$.
\end{denotation}

\begin{definition}\label{def:3.3:isogeny_height}
    Пусть $\sG$ - формальная группа размерности $m$ над $\fO_K$, ${ \varphi : \sF \rightarrow \sG }$ - изогения, ${ \Ker{\varphi} \cong \Spf{A} }$ для некоторой аугментированной нильпотентной $\fO_K$-алгебры $A$. Для любого простого идеала ${ \fp \subset \fO_K }$ пусть $\kappa(\fp)$ - его поле вычетов, тогда
    \begin{equation*}
        h(\fp) =
        \log_p \left(
            \dim_{\kappa(\fp)} A \otimes_{\fO_K} \kappa(\fp)
        \right) \in \bN\footnote{
            \cite{Zink}, замечание 5.6.
        }.
    \end{equation*}
    Если существует такое число ${ h \in \bN }$, что ${ h(\fp) = h }$ для всех ${ \fp \subset \fO_K }$, то это $h$ мы обозначим через $h(\varphi)$ и назовём \textit{высотой изогении $\varphi$}\footnote{
        Такого числа может не быть, и в этом случае высота изогении просто не определена.
    }.
\end{definition}

\begin{denotation}\label{denote:3.6:formal_group_height}
    Если для $\sF$ определена высота $h\left([p]_\sF\right)$ умножения на $p$, то это число мы обозначим через ${ h = h(\sF) \in \bN }$ и назовём \textit{высотой формальной группы $\sF$}.
\end{denotation}

\begin{denotation}\label{denote:3.4:multiplication_by_n}
    Для любого ${ n \in \bN }$ обозначим через ${ [n]_\sF \in \End{\sF} }$ эндоморфизм умножения на $n$. То есть, такой морфизм, что для любого ${ N \in \Nil{\fO_K} }$:
    \begin{equation*}
        \left([n]_\sF\right)_N :
        a \mapsto
        \underbrace{
            a +_{\sF(N)} ... +_{\sF(N)} a
        }_{\text{$n$ слагаемых}},
    \end{equation*}
    где ${ a \in \sF(N) }$, $+_{\sF(N)}$ - операция сложения в группе $\sF(N)$.
\end{denotation}

\begin{definition}\label{def:3.5:p-divisible_formal_group}
    Формальная группа $\sF$ называется \textit{$p$-делимой}, если $[p]_\sF$ является изогенией\footnote{
        Здесь, как и ранее, ${ p = \fchar{\overline{K}} > 0 }$.
    }.
\end{definition}

Примеры:
\begin{itemize}
    \item Формальная группа $\Ga^m$ над $\fO_K$ не является $p$-делимой\footnote{
        Проверить, что $\Ga^m$ не $p$-делима над $\overline{K}$ можно, например, с помощью \cite{Zink}, 5.27. Так как свойство $p$-делимости сохраняется при замене базы (\cite{Zink}, 5.6), то $\Ga^m$ не $p$-делима и над $\fO_K$.
    }.
    \item Формальная группа $\Gm^m$ $p$-делима над $\fO_K$\footnote{
        Вновь можно воспользоваться заменой базы. Над $\overline{K}$ $m$-мерный случай аналогичен одномерному, рассмотренному в \cite{Zink}, в начале \S 4 раздела 5.
    }.
    \item Формальная группа $\sF_{\Gamma, 1}$ $p$-делима над $\fO_K$\footnote{
        Переходим к $\overline{K}$. Для любого ${ \xi \in M_{\sF_{\Gamma, 1}} \cong \sC(F_\Gamma) }$:
        \begin{equation*}
            [p]_{\sF_{\Gamma, 1}}(\xi) =
            [p I_m]_\Gamma(\xi) =
            f_\Gamma^{-1}(p f_\Gamma(\xi)) =
            \xi^p + ...,
        \end{equation*}
        поэтому $[p]_{\sF_{\Gamma, 1}}$ инъективно на $M_{\sF_{\Gamma, 1}}$. Следовательно, по \cite{Zink}, 5.27, $\sF_{\Gamma, 1}$ $p$-делима над $\overline{K}$, а значит и над $\fO_K$.

        На самом деле, аналогичные рассуждения применимы и для $\sF_{\Gamma, h}$ при ${ h > 1 }$.
    }.
\end{itemize}

\begin{remark}\label{remark:3.after_p-divisible:sF}
   С этого момента под $\sF$ подразумевается некоторая $p$-делимая формальная группа размерности $m$ над $\fO_K$ (например, $\Gm^m$ или $\sF_{\Gamma, h}$).
\end{remark}

\begin{theorem}\label{theorem:3.3:exp_existence}
    Если формальная группа $\sF$ строго пропредставима, то существует единственныий функториальный по $\sF$\footnote{
        Подразумевается, что $\exp$ ведёт себя как функтор, то есть, для каждого гомоморфизма строго пропредставимых формальных групп ${ \varphi : \sF \rightarrow \sG }$ над $K$ определён также некоторый гомоморфизм ${ \Lie{\varphi} : \Lie{\sF} \rightarrow \Lie{\sG} }$, причём ${ \varphi \circ \exp_\sF = \exp_\sG \circ \Lie{\varphi} }$, ${ \Lie{\id_\sF} = \id_{\Lie{\sF}} }$ и ${ \Lie{\psi \circ \varphi} = \Lie{\psi} \circ \Lie{\varphi} }$.
    } изоморфизм ${ \exp_\sF = \exp_{\iota_* \sF} : \Lie{\iota_* \sF} \rightarrow \iota_* \sF }$, где ${ \iota : \fO_K \rightarrow K }$ - вложение.
\end{theorem}
\begin{proof}
    \cite{ZinkLectures}, предложение 1.38.
\end{proof}

\begin{remark}\label{remark:3.after_exp_existence:sF}
   Теперь рассматриваемая $p$-делимая $m$-мерная формальная группа $\sF$ над $\fO_K$ также должна быть строго пропредставимой (в качестве примеров всё ещё подходят $\Gm^m$ или $\sF_{\Gamma, h}$).
\end{remark}

\begin{denotation}\label{denote:3.after_exp_existence:log_sF}
    Пусть $\lambda_\sF$ - логарифм $\sF$, т.е. ${ \lambda_\sF = \exp_\sF^{-1} }$.
\end{denotation}

\end{subsection}

\begin{subsection}{Логарифм формальной группы представимый в виде ряда}

\begin{denotation}
    Для ${ (C, \{ \fc_i \}) \in \Compl{K} }$ пусть ${ \forget{K}^C : C \rightarrow \forget{K}(C) }$ - морфизм, который действует как тождественный на множестве $C$, если забыть про все структуры на нём. Мы также обозначим через ${ \forget{K}^{-C} : \forget{K}(C) \rightarrow C }$ обратный к $\forget{K}^C$ морфизм. Эти морфизмы сохраняют аддитивную структуру, но не мультипликативную.
\end{denotation}

\begin{denotation}
    Для ${ (C, \{ \fc_i \}) \in \Compl{K} }$ обозначим через ${ s_C : K \rightarrow C }$ структурный морфизм $K$-алгебры $C$, а через ${ \alpha_C : K \rightarrow \End{C} }$ - гомоморфизм, задающий структуру $K$-модуля на $C$\footnote{
        Другими словами, можно сказать, что $\alpha_C$ - представление $K$, ассоциированное с $C$ (см. \cite{Passman}).
    }. Таким образом, ${ \alpha_C(a)(f) = s_C(a) \cdot f }$ для всех ${ a \in K }$, ${ f \in C }$.

    Если ${ N \in \Nil{K} }$, то такой $K$-алгебре будут соответствовать структурный морфизм ${ s_N : K \rightarrow N }$ и гомоморфизм ${ \alpha_N : K \rightarrow \End{N} }$, согласованные с вложением категории $\Nil{K}$ в $\Compl{K}$.

    Если же ${ M \in \Mod{K} }$, то структурного морфизма мы не имеем, но можно также через ${ \alpha_M : K \rightarrow \End{M} }$ обозначить гомоморфизм, задающий структуру $K$-модуля на $M$. Тогда для ${ (C, \{ \fc_i \}) \in \Compl{K} }$ будет иметь место
    \begin{equation*}
        \alpha_{\forget{K}(C)}(a) \left(
            \forget{K}^C(f)
        \right) =
        \forget{K}^C \left(
            s_C(a) \cdot f
        \right),
        a \in K,
        f \in C.
    \end{equation*}
\end{denotation}

\begin{denotation}
    Пусть $\sF$ - формальная группа над $K$, ${ \sL = \Lie{\sF} }$. Для любого ${ (C, \{ \fc_i \}) \in \Compl{K} }$ обозначим через ${ \varphi_{\sF, C} : C \rightarrow \End{\sL(C)} }$ - гомоморфизм колец ${ \varphi_{\sF, C} : f \mapsto t_\sF([f]_C) }$, где ${ [f]_C : \forget{K}(C) \rightarrow \forget{K}(C) }$, ${ [f]_C : g \mapsto \forget{K}^C\left(f \cdot \forget{K}^{-C}(g)\right) }$. При этом, ${ [s_C(a)]_C = \alpha_{\forget{K}(C)}(a) }$ для всякого ${ a \in K }$. Получается, что гомоморфизм ${ \varphi_{\sF, C} }$ задаёт структуру $C$-модуля на $\sL(C)$.
\end{denotation}

\begin{definition}\label{def-end-translator}
    Пусть $\sF$ - формальная группа над $K$, ${ \sL = \Lie{\sF} }$. Предположим также, что $K$-модуль ${ \sL(K) = t_\sL(K) \cong M_\sL / M_\sL^2 }$\footnote{
        См. \cite{Zink}, 4.18 и 4.23.
    } свободен\footnote{
        Согласно \cite{Zink}, 4.7, это означает, что ${ \sL(C) \cong \bigoplus\limits_{i \in I} \forget{K}(C) }$ для всех ${ (C, \{ \fc_i \}) \in \Compl{K} }$.
    }, и что $\Cart(K)$-модуль ${ M_\sL }$ обладает стандартным $V$-базисом ${ \{ \delta_i \}_{i \in I} }$\footnote{
        То есть ${ (\delta_i)_j = \begin{cases} x, & \text{если ${ i = j }$,} \\ 0, & \text{иначе} \end{cases} }$, где ${ i, j \in I }$.
    } для некоторого (не обязательно конечного или даже счётного) множества индексов $I$. Тогда кольцо эндоморфизмов модуля ${ \sL(K) }$ можно представить в виде
    \begin{equation*}
        \End{\sL(K)} =
        M_I(K) =
        \left\{
            \xi : I \times I \rightarrow K
        \middle|
            \forall i \in I:
            \#\{ j \in I | \xi(i, j) \neq 0 \} < \infty
        \right\},
    \end{equation*}
    в котором операции сложения и умножения выглядят следующим образом:
    \begin{gather*}
        (\xi + \eta)(i, j) =
        \xi(i, j) + \eta(i, j), \\
        (\xi \eta)(i, j) =
        \sum_{k \in I} \xi(k, j) \eta(i, k).
    \end{gather*}
    На ${ \sL(K) }$ такие эндоморфизмы будут действовать по правилу
    \begin{equation*}
        \xi :
        (a_i)_{i \in I} \mapsto
        \left(
            \sum_{j \in I} \alpha_K(\xi(j, i))(a_j)
        \right)_{i \in I}.
    \end{equation*}
    Определим отображение ${ \sigma_{K[[x]]^+} : \End{\sL(K)} \rightarrow \End{M_\sL} }$ так:
    \begin{equation*}
        \sigma_{K[[x]]^+}(\xi) :
        \delta_i \mapsto
        \sum_{j \in I} \oH{\xi(i, j)} \delta_j\footnote{
            Так как любой элемент из ${ M_\sL }$ представим через элементы $V$-базиса, то достаточно определить действие гомоморфизма ${ \sigma_{K[[x]]^+}(\xi) }$ на ${ \{ \delta_i \}_{i \in I} }$
        }.
    \end{equation*}
    Корректность определения связана со свойствами ${ \xi \in M_I(K) }$.

    Далее, рассмотрим произвольный ${ N \in \Nil{K} }$. Следующим шагом будет определение отображения ${ \sigma_N : \End{\sL(K)} \rightarrow \End{\sL(N)} }$. Пусть ${ \xi \in \End{\sL(K)} }$, ${ \nu \in \sL(N) }$, и элементу $\nu$ соответствует пара ${ (u, \gamma) }$ относительно изоморфизма ${ \sL(N) \cong \Lambda(N) \overline{\otimes}_{\Cart(K)} M_\sL }$\footnote{
        См. \cite{Zink}, 3.28.
    }, где ${ u = 1 + u_1 t + ... + u_n t^n \in \Lambda(N) }$, ${ \gamma \in M_\sL }$. Пусть ${ v^n_i : K[[x]]^+ \rightarrow K[[X_1, ..., X_n]]^+ }$ - гомоморфизм вида ${ v^n_i : x \mapsto X_i }$, и ${ v^n_\sL = \sum\limits_{i = 1}^n \sL (v^n_i) : M_\sL \rightarrow \sL(K[[X_1, ..., X_n]]^+) }$. Пусть ${ \rho^u_n : K[[X_1, ..., X_n]]^+ \rightarrow N }$ - гомоморфизм вида ${ \rho^u_n : X_i \mapsto (-1)^i u_i }$. Тогда положим
    \begin{equation*}
        \sigma_N(\xi) :
        \nu \mapsto
        \sL(\rho^u_n) \left(
            v^n_\sL \left(
                \sigma_{K[[x]]^+}(\xi)(\gamma)
            \right)
        \right).
    \end{equation*}

    Таким образом, зафиксировав некоторый $V$-базис $\Cart(K)$-модуля $M_\sL$, мы можем построить по нему систему морфизмов ${ \sigma = \{ \sigma_N \}_{N \in \Nil{K}} }$, которую назовём \textit{переносом эндоморфизмов над $\Nil{K}$ для $\sF$}. Также иногда будем говорить, что \textit{$\sF$ допускает перенос эндоморфизмов над $\Nil{K}$ посредством $\sigma$}\footnote{
        Другими словами, если мы говорим, что $\sF$ допускает перенос эндоморфизмов посредством $\sigma$, то мы подразумеваем, что $M_\sL$ обладает стандартным $V$-базисом, и что $\sigma$ строится с помощью этого базиса так, как это указано в данном определении.
    }.
\end{definition}

\begin{claim}\label{claim-sigmaN_property}
    Пусть $\sF$ - формальная группа над $K$, ${ \sL = \Lie{\sF} }$, и $\sF$ допускает перенос эндоморфизмов над $\Nil{K}$ посредством ${ \sigma = \{ \sigma_N \}_{N \in \Nil{K}} }$. Пусть ${ N, M \in \Nil{K} }$, и ${ f \in \mathrm{Hom}_{\Mod{K}}(\forget{K}(N), \forget{K}(M)) }$ - гомоморфизм $K$-модулей. Тогда, ${ \forall \xi \in \End{\sL(K)} }$:
    \begin{center}
        \begin{tikzcd}
            \sL(N)
                \arrow{r}{\sigma_N(\xi)}
                \arrow[swap]{d}{t_\sF(f)} &
            \sL(N)
                \arrow{d}{t_\sF(f)} \\
            \sL(M)
                \arrow{r}{\sigma_M(\xi)} &
            \sL(M)
        \end{tikzcd}
    \end{center}
    - коммутативная диаграмма.
\end{claim}
\begin{proof}
    Из \cite{Zink} известно, что изоморфизм ${ \Lambda(N) \overline{\otimes}_{\Cart(K)} M_\sL \cong \sL(N) }$ действует по правилу ${ (u, \gamma) \mapsto \sL(\rho^u_n)(v_\sL^n(\gamma)) }$\footnote{
        См. \cite{Zink}, 3.28. Обозначения взяты из определения \ref{def-end-translator}.
    }. Используя этот факт, а также определения $\sL$, $\sigma_N$ и свойства функторов, не трудно убедиться в истинности утверждения.
\end{proof}

\begin{definition}
    Благодаря \ref{claim-sigmaN_property}, мы можем дополнить определение \ref{def-end-translator}.

    Пусть $\sF$ - формальная группа над $K$, ${ \sL = \Lie{\sF} }$, и $\sF$ допускает перенос эндоморфизмов над $\Nil{K}$ посредством ${ \sigma = \{ \sigma_N \}_{N \in \Nil{K}} }$.

    Пусть ${ (C, \{ \fc_i \}) \in \Compl{K} }$. Определим отображение ${ \sigma_C : \End{\sL(K)} \rightarrow \End{\sL(C)} }$. Пусть ${ \xi \in \End{\sL(K)} }$, ${ \varphi \in \sL(C) }$. По определению $\sL(C)$, $\varphi$ имеет вид ${ \varphi = (\varphi_i)_{i = 1}^\infty }$, где ${ \varphi_i \in \sL(\fc_1 / \fc_i) }$ и ${ \varphi_i = \sL(f_{i j})(\varphi_j) }$ при ${ i \leq j }$, и ${ f_{i j} : \fc_1 / \fc_j \rightarrow \fc_1 / \fc_i }$, ${ f_{i j} : f \mod \fc_j \mapsto f \mod \fc_i }$. Тогда положим
    \begin{equation*}
        \sigma_C(\xi)(\varphi) =
        \left(
            \sigma_{\fc_1 / \fc_i}(\xi)(\varphi_i)
        \right)_{i = 1}^\infty.
    \end{equation*}
    Именно корректность этого определения проверяется с помощью \ref{claim-sigmaN_property}.

    Теперь систему морфизмов ${ \sigma = \{ \sigma_C \}_{(C, \{ \fc_i \}) \in \Compl{K}} }$ мы можем назвать \textit{переносом эндоморфизмов для $\sF$}, и иногда говорить, что \textit{$\sF$ допускает перенос эндоморфизмов посредством $\sigma$}.
\end{definition}

\begin{claim}\label{claim-sigmaC_properties}
    Пусть $\sF$ - формальная группа над $K$, ${ \sL = \Lie{\sF} }$, и $\sF$ допускает перенос эндоморфизмов посредством ${ \sigma = \{ \sigma_C \}_{(C, \{ \fc_i \}) \in \Compl{K}} }$. Тогда
    \begin{itemize}
        \item Пусть ${ C, D \in \Compl{K} }$, и ${ f \in \mathrm{Hom}_{\Mod{K}}(\forget{K}(C), \forget{K}(D)) }$ - гомоморфизм $K$-модулей. Тогда, ${ \forall \xi \in \End{\sL(K)} }$:
        \begin{center}
            \begin{tikzcd}
                \sL(C)
                    \arrow{r}{\sigma_C(\xi)}
                    \arrow[swap]{d}{t_\sF(f)} &
                \sL(C)
                    \arrow{d}{t_\sF(f)} \\
                \sL(D)
                    \arrow{r}{\sigma_D(\xi)} &
                \sL(D)
            \end{tikzcd}
        \end{center}
        - коммутативная диаграмма,
        \item ${ \forall (C, \{ \fc_i \}) \in \Compl{K} }$: ${ \sigma_C : \End{\sL(K)} \rightarrow \End{\sL(C)} }$ - гомоморфизм колец,
        \item ${ \forall (C, \{ \fc_i \}) \in \Compl{K} }$: ${ \sigma_C \circ \varphi_{\sF, K} = \varphi_{\sF, C} \circ s_C }$.
    \end{itemize}
\end{claim}
\begin{proof}
    Первое свойство проверяется с помощью \ref{claim-sigmaN_property}.

    Чтобы проверить второе свойство, достаточно использовать определение и уже доказанное свойство $\sigma_C$, свойства проективных пределов и тот факт, что $\sigma_N$ является гомоморфизмом колец для любого ${ N \in \Nil{K} }$. А это, в свою очередь, следует из определения $\sigma_N$, и того, что для любых ${ \xi, \eta \in \sL(K) }$ и ${ \gamma \in M_\sL }$:
    \begin{gather*}
        \sigma_{K[[x]]^+}(\xi + \eta)(\gamma) =
        \sigma_{K[[x]]^+}(\xi)(\gamma) +
        \sigma_{K[[x]]^+}(\eta)(\gamma), \\
        \sigma_{K[[x]]^+}(\xi \eta)(\gamma) =
        \sigma_{K[[x]]^+}(\xi)\left(
            \sigma_{K[[x]]^+}(\eta)(\gamma)
        \right).
    \end{gather*}
    Для проверки этих утверждений, можно ограничиться случаем ${ \gamma = \delta_i }$, где ${ \{ \delta_i \}_{i \in I} }$ - стандартный $V$-базис $M_\sL$. Дальше надо использовать определение $\sigma_{K[[x]]^+}$, а также свойство ${ \oH{a + b} \delta_i = \oH{a} \delta_i + \oH{b} \delta_i }$ для любых ${ a, b \in K }$ - именно из-за этого свойства предпочтительно брать стандартный базис $M_\sL$, а не любой другой, иначе пришлось бы вносить корректировки в определение $\sigma_{K[[x]]^+}$ и в обоснование данного свойства.

    Докажем третье свойство. Используя определения $\sigma_C$ и $\varphi_{\sF, C}$, можно свести его к случаю, когда ${ C = N \in \Nil{K} }$. Далее, используя разложение элементов из $M_\sL$ по $V$-базису, можно ограничиться доказательством того факта, что
    \begin{equation*}
        \sL(\rho^u_n)\left(
            v_\sL^n\left(
                \sigma_{K[[x]]^+}\left(
                    \varphi_{\sF, K}(a)
                \right)(V_r \oH{b} \delta_i)
            \right)
        \right) =
        \varphi_{\sF, N}(s_N(a))(\nu),
    \end{equation*}
    где элементу ${ \nu \in \sL(N) }$ соответствует пара ${ (u, V_r \oH{b} \delta_i) }$ относительно изоморфизма ${ \sL(N) \cong \Lambda(N) \overline{\times}_{\Cart(K)} M_\sL }$, где ${ r \geq 1 }$, ${ i \in I }$, ${ a, b \in K }$, ${ u \in \Lambda(N) }$, ${ \deg(u) = n }$. На самом деле, проверить это не сложно - все так же пользуясь лишь определениями и уже известными свойствами $\sigma_{K[[x]]^+}$, $\varphi_{\sF, N}$, $\rho^u_n$ и $v_\sL^n$.
\end{proof}

\begin{claim}\label{claim-sigmaC_sufficient}
    Пусть $\sF$ - формальная группа над $K$, ${ \sL = \Lie{\sF} }$. Пусть ${ \sigma = \{ \sigma_C : \End{\sL(K)} \rightarrow \End{\sL(C)} \}_{(C, \{\fc_i\}) \in \Compl{K}} }$ - набор гомоморфизмов колец, такой, что:
    \begin{itemize}
        \item Для любого ${ \xi \in \End{\sL(K)} }$
        \begin{equation*}
            \sigma_{K[[x]]^+}(\xi) :
            \delta_i \mapsto
            \sum_{j \in I} \oH{\xi(i, j)} \delta_j.
        \end{equation*}
        \item Пусть ${ C, D \in \Compl{K} }$, и ${ f \in \mathrm{Hom}_{\Mod{K}}(\forget{K}(C), \forget{K}(D)) }$ - гомоморфизм $K$-модулей. Тогда, ${ \forall \xi \in \End{\sL(K)} }$:
        \begin{center}
            \begin{tikzcd}
                \sL(C)
                    \arrow{r}{\sigma_C(\xi)}
                    \arrow[swap]{d}{t_\sF(f)} &
                \sL(C)
                    \arrow{d}{t_\sF(f)} \\
                \sL(D)
                    \arrow{r}{\sigma_D(\xi)} &
                \sL(D)
            \end{tikzcd}
        \end{center}
        - коммутативная диаграмма.
    \end{itemize}
    Тогда система $\sigma$ является переносом эндоморфизмов для $\sF$.
\end{claim}
\begin{proof}
    Доказывается просто.
\end{proof}

\begin{remark}
   Утверждение \ref{claim-sigmaC_properties} даёт необходимые, а \ref{claim-sigmaC_sufficient} - достаточные условия для того, чтобы система гомоморфизмов $\sigma$ была переносом эндоморфизмов для формальной группы $\sF$.
\end{remark}

\begin{claim}\label{claim-end_traslation_for_finite_F}
    Пусть $\sF$ - конечномерная формальная группа размерности $m$ над $K$, ${ \sL = \Lie{\sF} }$. Тогда ${ \End{\sL(C)} = M_m\left(\End{\forget{K}(C)}\right) }$ для любого ${ (C, \{\fc_i\}) \in \Compl{K} }$, ${ \End{\sL(K)} = M_m(K) }$, и $\sF$ допускает перенос эндоморфизмов посредством ${ \sigma = \{ \sigma_C \}_{(C, \{ \fc_i \}) \in \Compl{K}} }$, где
    \begin{equation*}
        \sigma_C :
        \left(
            \begin{matrix}
                a_{1 1} & \hdots & a_{1 m} \\
                \vdots & & \vdots \\
                a_{m 1} & \hdots & a_{m m}
            \end{matrix}
        \right) \mapsto
        \left(
            \begin{matrix}
                \alpha_{\forget{K}(C)}(a_{1 1}) &
                \hdots &
                \alpha_{\forget{K}(C)}(a_{1 m}) \\
                \vdots & & \vdots \\
                \alpha_{\forget{K}(C)}(a_{m 1}) &
                \hdots &
                \alpha_{\forget{K}(C)}(a_{m m})
            \end{matrix}
        \right)
    \end{equation*}
\end{claim}
\begin{proof}
    Кольцо $\End{\sL(C)}$ принимает указанный вид потому, что ${ \sL \cong \Ga^m }$, согласно теореме 1.39 из \cite{ZinkLectures}.

    Для доказательства проще всего воспользоваться критерием \ref{claim-sigmaC_sufficient}. Свойства матриц и $\alpha_{\forget{K}(C)}$ влекут тот факт, что $\sigma_C$ - гомоморфизм колец. Остальные условия также проверяются напрямую по определению.
\end{proof}

\begin{denotation}
    Пусть ${ \sF \cong \Spf{C, \{\fc_k\}} }$ - пропредставимая формальная группа над полем $K$, ${ (C, \{ \fc_k \}) \in \Compl{K} }$. Обозначим изоморфизм ${ \sF \rightarrow \Spf{C, \{\fc_k\}} }$ через $\forget{\sF, C}$.
\end{denotation}

\begin{denotation}\label{denote-Delta_sF}
    Пусть ${ \sF \cong \Spf{C, \{\fc_k\}} }$ - строго пропредставимая формальная группа над полем $K$, ${ (C, \{ \fc_k \}) \in \Compl{K} }$. Пусть ${ \{ \varepsilon_i \}_{i \in I} }$\footnote{
        ${ |I| < \infty }$, согласно определению строгой пропредставимости.
    } - множество элементов из $\fc_1$, таких, что ${ \{ \varepsilon_i \mod \fc_2 \}_{i \in I} }$ - базис модуля ${ \fc_1 / \fc_2 }$. Пусть ${ P_C = K\langle\langle Z_i \rangle\rangle_{i \in I} }$ - $K$-алгебра формальных степенных рядов от некоммутирующих переменных $Z_i$, ${ \alpha_C : P_C \rightarrow C }$ - сюръективный гомоморфизм\footnote{
        см. \cite{Quillen}, следствие A.1.7.
    }, отправляющий $Z_i$ в $\varepsilon_i$. Пусть ${ \alpha_{C, k} : \alpha_C^{-1}(\fc_1) / \alpha_C^{-1}(\fc_k) \rightarrow \fc_1 / \fc_k }$ - изоморфизм $K$-алгебр, действующий следующим образом: ${ \alpha_{C, k} : \overline{s} \mapsto \overline{\alpha_C(s)} }$, где черта означает переход к классу смежности по модулю соответствующего идеала. Для ${ N \in \Nil{K} }$, пусть ${ \tau_{C, k, N} : \Hom{\fc_1 / \fc_k}{N} \rightarrow \Hom{\alpha_C^{-1}(\fc_1) / \alpha_C^{-1}(\fc_k)}{N} }$ - биекция, такая, что ${ \tau_{C, k, N} : f \mapsto f \circ \alpha_{C, k} }$. Для всякого натурального ${ n > 0 }$ пусть ${ d_{C, k}^n \in \mathrm{End}_{\text{$K$-alg}}\left(\alpha_C^{-1}(\fc_1) / \alpha_C^{-1}(\fc_k)\right) }$ определяется следующим образом: ${ d_{C, k}^n : \overline{s(Z_i)_{i \in I}} \mapsto \overline{s(Z_i^n)_{i \in I}} }$, то есть, $d_{C, k}^n$ заменяет каждый ${ Z_i }$ на ${ Z_i^n }$ в элементах из ${ \alpha_C^{-1}(\fc_1) / \alpha_C^{-1}(\fc_k) }$. Тогда для того же ${ n > 0 }$ пусть ${ d_{C, k, N}^n : \Hom{\alpha_C^{-1}(\fc_1) / \alpha_C^{-1}(\fc_k)}{N} \rightarrow \Hom{\alpha_C^{-1}(\fc_1) / \alpha_C^{-1}(\fc_k)}{N} }$ - отображение вида ${ d_{C, k, N}^n : s \mapsto s \circ d_{C, k}^n }$. Это приводит к корректному определению отображения ${ D_{C, k, N}^n = \tau_{C, k, N}^{-1} \circ d_{C, k, N}^n \circ \tau_{C, k, N} }$, действующего на множестве ${ \Hom{\fc_1 / \fc_k}{N} }$. Так как такие отображения коммутируют с морфизмами направленной системы ${ \left\{ \Hom{\fc_1 / \fc_k}{N} \right\}_{k > 0} }$, то по ним можно однозначно восстановить отображение между индуктивными пределами ${ D_{C, N}^n : \Spf{C, \{\fc_k\}}(N) \rightarrow \Spf{C, \{\fc_k\}}(N) }$. В таком случае, обозначим ${ D_{\sF(N)}^n = (\forget{\sF, C})_N^{-1} \circ D_{C, N}^n \circ (\forget{\sF, C})_N : \sF(N) \rightarrow \sF(N) }$, и пусть ${ \Delta_{\sF(N)} = D_{\sF(N)}^p }$, где по-прежнему ${ p = \fchar{\overline{K}} > 0 }$. Дополнительно обозначим ${ \Delta_{\sF(N)}^0 = \id_{\sF(N)} }$ и ${ \Delta_{\sF(N)}^n = \Delta_{\sF(N)} \circ \Delta_{\sF(N)}^{n - 1} }$ для любого целого ${ n > 0 }$.
\end{denotation}

\begin{remark}
   В \ref{denote-Delta_sF} множество раз встаёт вопрос корректности, существования и единственности тех или иных обозначений и определений. Однако, по отдельности они проверяются относительно легко и не требуют подробных разъяснений.
\end{remark}

\begin{remark}
   Итоговое отображение ${ \Delta_{\sF(N)} }$ из \ref{denote-Delta_sF}, в общем случае, не обязано быть эндоморфизмом абелевой группы $\sF(N)$. Его надо воспринимать именно как отображение множества в себя.
\end{remark}

\begin{claim}\label{claim-Delta_sF_property}
    Система отображений ${ \{\Delta_{\sF(N)}\}_{N \in \Nil{K}} }$ из \ref{denote-Delta_sF} обладает следующим свойством: для любых ${ N, M \in \Nil{K} }$ и любого ${ f \in \Hom{N}{M} }$ коммутативна следующая диаграмма:

    \begin{center}
        \begin{tikzcd}
            \sF(N)
                \arrow{r}{\Delta_{\sF(N)}}
                \arrow[swap]{d}{\sF(f)} &
            \sF(N)
                \arrow{d}{\sF(f)} \\
            \sF(M)
                \arrow{r}{\Delta_{\sF(M)}} &
            \sF(M)
        \end{tikzcd}
    \end{center}
    
    Следовательно, ${ \Delta_\sF = \{\Delta_{\sF(N)}\}_{N \in \Nil{K}} }$ является естественным преобразованием функтора $\sF$, если рассматривать его как функтор из $\Nil{K}$ в $\Sets$.
\end{claim}
\begin{proof}
    Без труда проверяется по определению.
\end{proof}

\begin{claim}\label{claim-Delta_sF_property2}
    Пусть $\sF$ - строго пропредставимая формальная группа над полем $K$, ${ N \in \Nil{K} }$. Тогда для любого ${ \nu \in \sF(N) }$ существует некоторое целое ${ n \geq 0 }$, такое, что ${ \Delta_{\sF(N)}^n(\nu) = 0 }$.
\end{claim}
\begin{proof}
    Доказывается не сложно, надо лишь использовать нильпотентность $N$ и строгую пропредставимость $\sF$.
\end{proof}

\begin{claim}\label{claim-Delta_sF_for_finite_sF}
    Пусть $\sF$ - конечномерная формальная группа размерности $m$ над $K$. Тогда ${ \Delta_{\sF(N)} : \sF(N) \rightarrow \sF(N) }$ принимает вид ${ \Delta_{\sF(N)} : (\nu_1, ..., \nu_m) \mapsto (\nu_1^p, ..., \nu_m^p) }$.
\end{claim}
\begin{proof}
    Известно, что ${ \sF \cong \Spf{C, \{\fc_k\}} }$, где ${ C = K[[X_1, ..., X_m]] }$, и ${ \fc_k = (X_1, ..., X_m)^k }$. Используя этот факт и определение \ref{denote-Delta_sF}, можно непосредственно получить требуемое.
\end{proof}

\begin{denotation}\label{claim-forget_sF-sL}
    Пусть $\sF$, $\sG$ - пропредставимые формальные группы над полем $K$, такие, что ${ \sF \cong \sG \cong \Spf{C, \{\fc_k\}} }$ для некоторого ${ (C, \{ \fc_k \}) \in \Compl{K} }$. Введём обозначение ${ \forget{\sF, \sG} = \forget{\sG, C}^{-1} \circ \forget{\sF, C} }$.
\end{denotation}

\begin{remark}
    Изоморфизм $\forget{\sF, \sG}$ из \ref{claim-forget_sF-sL} в общем случае не является изоморфизмом формальных групп. Это лишь изоморфизм двух функторов из $\Nil{K}$ в $\Sets$.
\end{remark}

\begin{claim}\label{claim-forget_sF-sL_for_finite_sF}
    Пусть $\sF$ - конечномерная формальная группа размерности $m$ над $K$, ${ \sL = \Lie{\sF} }$. Тогда для любого ${ N \in \Nil{K} }$ биекция ${ \left(\forget{\sF, \sL}\right)_N : \sF(N) \rightarrow \sL(N) }$ действует следующим образом: ${ (\nu_1, ..., \nu_m) \mapsto \left(\forget{K}^N(\nu_1), ..., \forget{K}^N(\nu_m)\right) \in \left( \forget{K}(N)^m, +_{t_\sF} \right) = (N^m, +_\sL) }$.
\end{claim}
\begin{proof}
    Легко следует из определения.
\end{proof}

\begin{definition}
    Пусть $\sF$, $\sG$ - строго пропредставимые формальные группы над полем $K$, ${ \sL = \Lie{\sG} \cong \Lie{\sF} }$, и пусть $\sG$ допускает перенос эндоморфизмов посредством $\sigma$. Пусть ${ \Lambda = \sum\limits_{k = 0}^\infty \Lambda_k \Delta^k \in \End{\sL(K)}[[\Delta]] }$. Обозначим через $\nat{\Lambda}{\sF}{\sG}$ естественное преобразование, соответствующее $\Lambda$, переводящее $\sF$ в $\sG$, которые рассматриваются как функторы из $\Nil{K}$ в $\Sets$. Действие $\nat{\Lambda}{\sF}{\sG}$ определим следующим образом:
    \begin{equation*}
        \nat{\Lambda}{\sF}{\sG}_N :
        \nu \mapsto
        (\forget{\sL, \sG})_N \left(
            \sum_{k = 0}^\infty
            \sigma_N(\Lambda_k) \left(
                (\forget{\sF, \sL})_N \left(
                    \Delta_{\sF(N)}^k(\nu)
                \right)
            \right)
        \right),
    \end{equation*}
    где ${ \nu \in \sF(N) }$, и суммирование ведётся с помощью операции ${ +_\sL }$ в ${ \sL(N) }$.
    Корректность определения\footnote{
        Все морфизмы применимы, сумма конечна, и набор ${ \{ \nat{\Lambda}{\sF}{\sG}_N \}_{N \in \Nil{K}} }$ действительно определяет естественное преобразование функторов, т.е. ${ \nat{\Lambda}{\sF}{\sG}_M \circ \sF(f) = \sG(f) \circ \nat{\Lambda}{\sF}{\sG}_N }$ для любого гомоморфизма $K$-алгебр ${ f \in \Hom{N}{M} }$.
    } вытекает из утверждений \ref{claim-Delta_sF_property}, \ref{claim-Delta_sF_property2} и свойств $\sigma$.
\end{definition}

\begin{claim}\label{claim-p-typical_criterion_through_Lambda}
    Пусть $\sF$ - конечномерная формальная группа размерности $m$ над $\fO_K$, ${ \sL = \Lie{\iota_* \sF} }$, а $F$ - соответствующий ей групповой закон. Тогда групповой закон $F$ является $p$-типическим в том и только том случае, когда существует ряд ${ \Lambda \in \End{\sL(K)}[[\Delta]] }$, определяющий логарифм $\sF$, то есть, когда ${ \lambda_\sF = \nat{\Lambda}{\iota_* \sF}{\sL} }$.
\end{claim}
\begin{proof}
    Следует из \ref{claim:3.5:group_and_law_homs_correspondence}, \ref{claim-end_traslation_for_finite_F}, \ref{claim-Delta_sF_for_finite_sF} и \ref{claim-forget_sF-sL_for_finite_sF}.
\end{proof}

\begin{remark}
    \ref{claim-p-typical_criterion_through_Lambda} означает, что в некоторых случаях логарифм формальной группы полностью определяется некоторым рядом ${ \Lambda \in \End{\sL(K)}[[\Delta]] }$. Именно такие формальные группы будут интересовать нас далее.
\end{remark}

\begin{definition}
    Пусть $\sF$ - строго пропредставимая формальная группа над $\fO_K$, допускающая перенос эндоморфизмов, ${ \sL = \Lie{\iota_* \sF} }$. Если существует ряд ${ \Lambda \in \End{\sL(K)}[[\Delta]] }$, такой, что ${ \lambda_\sF = \nat{\Lambda}{\iota_* \sF}{\sL} }$, то мы будем говорить, что \textit{$\Lambda$ соответствуют формальной группе $\sF$}.
\end{definition}

\begin{guess}
    Пусть $\widehat{W}$ - формальная группа векторов Витта над $K$\footnote{
        ${ \widehat{W} : N \mapsto \Lambda(N) \varepsilon_1 }$ для любого ${ N \in \Nil{K} }$, где ${ \varepsilon_1 = \prod\limits_{\ell \neq p} \left( 1 - \dfrac{1}{\ell} \oV_\ell \oF_\ell \right) }$, ${ p = \fchar{\overline{K}} > 0 }$.
    }. Тогда существует ряд ${ \Lambda \in \End{\Lie{\sF}(K)}[[\Delta]] }$, соответствующий $\widehat{W}$\footnote{
        Это пример, выходящий за пределы утверждения \ref{claim-p-typical_criterion_through_Lambda}, поскольку формальная группа $\widehat{W}$ бесконечномерна.
    }.
\end{guess}

\begin{guess}
    Пусть $\sF$ - строго пропредставимая формальная группа над $\fO_K$, ${ \sL = \Lie{\iota_* \sF} \cong \bigoplus\limits_{i \in I} \Ga }$ для некоторого множества индексов $I$, и $M_\sL$ содержит только $p$-типические элементы\footnote{
        ${ p = \fchar{\overline{K}} > 0 }$.

        Возможно, это примерно то же самое, что требовать от формальной группы равенства ${ \sF = \widehat{\sF} }$, где ${ \widehat{\sF} : \Nil{K} \rightarrow \Ab }$, ${ \widehat{\sF} : N \mapsto \sF(N) \varepsilon_1 }$.

        Более точным могло бы быть требование, чтобы все элементы $M_\sL$ имели вид ${ \sum\limits_{k \geq 0} \sum\limits_{i \in I} V_{p^k} \oH{a_{k, i}} \delta_i }$, где ${ \{ \delta_i \}_{i \in I} }$ - стандартный базис $M_\sL$.
    }. Тогда существует ряд ${ \Lambda \in \End{\sL(K)}[[\Delta]] }$, соответствующий формальной группе $\sF$.
\end{guess}

\begin{denotation}
    Пусть $I$ - множество индексов, ${ p = \fchar{\overline{K}} }$. Положим
    \begin{equation*}
        p_I =
        \sum_{k = 0}^\infty p_{I, k} \Delta^k \in
        M_I(K)[[\Delta]],
    \end{equation*}
    где
    \begin{gather*}
        p_{I, 0} : (i, j) \mapsto \begin{cases}
            p, & i = j, \\
            0, & i \neq j;
        \end{cases} \\
        p_{I, k} : (i, j) \mapsto 0, k > 0.
    \end{gather*}
\end{denotation}

\begin{definition}
    Пусть $M$ - $V$-плоский $\Cart(K)$-модуль. Предположим, что существует такое множество индексов $I$, что ${ \iota : M \rightarrow \bigoplus\limits_{i \in I} \Ga(K[[x]]^+) }$ - изоморфизм $\Cart(K)$-модулей. Пусть $\{ \delta_i \}_{i \in I}$ - стандартный $V$-базис для ${ \bigoplus\limits_{i \in I} \Ga(K[[x]]^+) }$. Тогда мы назовём $V$-базис ${ \{ \delta^M_i = \iota^{-1}(\delta_i) \}_{i \in I} }$ \textit{стандартным $V$-базисом для $M$}.
\end{definition}

\begin{definition}
    Пусть $M$ и $N$ - $V$-плоские $\Cart(K)$-модули, обладающие стандартными $V$-базисами ${ \{ \delta^M_i \}_{i \in I} }$ и ${ \{ \delta^N_i \}_{i \in I} }$ для одного и того же множества индексов $I$. Пусть ${ \Lambda = \sum\limits_{k = 0}^\infty \Lambda_k \Delta^k \in M_I(K)[[\Delta]] }$. Обозначим через ${ \modhom{\Lambda}{M}{N} : M \rightarrow N }$ гомоморфизм модулей, соотвтетсвующий $\Lambda$, действующий следующим образом:
    \begin{equation*}
        \modhom{\Lambda}{M}{N} :
        \delta^M_i \mapsto
        \sum_{k = 0}^\infty
        \sum_{j \in I}
        \oH{\Lambda_k(i, j)} \oV_{p^k} \delta^N_j.
    \end{equation*}
    Так как ${ N = \varprojlim\limits_k N / V_{p^k} N }$ и ${ \#\{ j \in I | \Lambda_k(i, j) \neq 0 \} < \infty }$, ${ \forall i \in I }$, ${ k \geq 0 }$, то сумма в правой части формулы всегда конечна, поэтому гомоморфизм определён корректно.
\end{definition}

\begin{guess}
    Пусть $\sF$ - строго пропредставимая формальная группа над $\fO_K$, допускающая перенос эндоморфизмов, ${ \sL = \Lie{\iota_* \sF} }$, и ${ \Lambda \in \End{\sL(K)}[[\Delta]] }$ соответствует $\sF$. Тогда ${ \left( \lambda_\sF \right)_{K[[x]]^+} = \modhom{\Lambda}{M_{\iota_* \sF}}{M_\sL} }$.
\end{guess}

\end{subsection}

\end{section}

\newpage

\begin{section}{Инвариант дробной части r(F)}

\begin{theorem}\label{theorem:4.0:Lambda_as_fraction}
    Пусть ${ \Lambda_F \in M_m(K[[\Delta]]) }$ соответствует формальному групповому закону $F$ (т.е. ${ \Lambda_F(X) }$ является логарифмом $F$), ${ \Lambda_G }$ соответствует формальному групповому закону $G$ над $\fO_N$, при этом редукции $F$ и $G$ равны. Тогда $\Lambda_F$ можно представить в виде ${ v u^{-1} }$, где ${ v p^l \pi^{-p^l} \in M_m(\fO_K[[\Delta]]) }$, ${ l = \left\lfloor \log_p\left( \dfrac{e}{p - 1} \right) \right\rfloor }$, а ${ u = p \Lambda_G^{-1} \in M_m(\fO_{N, \sigma}[[\Delta]]) }$.
\end{theorem}
\begin{proof}
    См. \cite{BondarkoThesis}, теорема 4.2.1.
\end{proof}

\begin{remark}\label{remark:4.after_Lambda_as_fraction:incidence}
    Тот факт, что ${ p \Lambda_G^{-1} \in M_m(\fO_{N, \sigma}[[\Delta]]) }$, следует из доказательства теоремы 4.2.1 в \cite{BondarkoThesis}.
\end{remark}

\begin{remark}\label{remark:4.after_Lambda_as_fraction:calculation}
   На практике, ${ p \Lambda_G^{-1} }$ можно вычислять с помощью доказательства теормы, используя также \cite{Hazewinkel}, 20.3. Кроме того, можно попытаться вычислить ${ \Lambda_G^{-1}(X) }$ непосредственно, используя \cite{Hazewinkel}, A.4.6.
\end{remark}

Примеры:
\begin{itemize}
    \item Пусть ${ K = \bQ_p(\zeta_{p^n}) }$, ${ N = \bQ_p }$, ${ F = \GGa^m }$, ${ G = \GGa^m }$. Тогда ${ \Lambda_F = I_m }$, ${ \Lambda_G = I_m }$, ${ u = p I_m }$, ${ v = p I_m }$.
\end{itemize}

\begin{guess}
    Пусть ${ \iota : \fO_K \rightarrow K }$, ${ \omega : \fO_N \rightarrow N }$ и ${ \alpha : N \rightarrow K }$ - вложения. Пусть $\sF$ - строго пропредставимая $p$-делимая формальная группа над $\fO_K$, ${ \Lie{\iota_* \sF} \cong \bigoplus\limits_{i \in I} \Ga }$ и ${ \Lambda_\sF \in M_I(K)[[\Delta]] }$ соответствует $\sF$. Пусть $\sG$ - строго пропредставимая $p$-делимая формальная группа над $\fO_N$, ${ \Lie{\omega_* \sG} \cong \bigoplus\limits_{j \in J} \Ga }$ и ${ \Lambda_\sG \in M_J(N)[[\Delta]] }$ соответствует $\sG$. Предположим также, что ${ \overline{\sF} = \overline{\sG} }$. Тогда
    \begin{enumerate}
        \item ${ I \simeq J }$; это позволяет ввести обозначение ${ [p_I] = \nat{p_I}{(\alpha \circ \omega)_* \sG}{\iota_* \sF} }$;
        \item диаграмма
        \begin{center}
            \begin{tikzcd}
                (\alpha \circ \omega)_* \sG
                    \arrow{r}{\alpha_* \lambda_\sG}
                    \arrow[swap]{d}{[p_I]} &
                \Lie{(\alpha \circ \omega)_* \sG}
                    \arrow{d}{\Lie{[p_I]}} \\
                \iota_* \sF
                    \arrow{r}{\lambda_\sF} &
                \Lie{\iota_* \sF}
            \end{tikzcd}
        \end{center}
        коммутативна;
        \item $[p_I]$ и $\Lie{[p_I]}$ - изогении формальных групп;
        \item существует такой ряд ${ u \in M_I(\fO_N)[[\Delta]] }$, что ${ [p_I] \circ \left(\alpha_* \lambda_\sG\right)^{-1} = \nat{u}{\Lie{(\alpha \circ \omega)_* \sG}}{\iota_* \sF} }$;
        \item существует такой ряд ${ v \in M_I(K)[[\Delta]] }$, что ${ \Lie{[p_I]} = \nat{v}{\Lie{(\alpha \circ \omega)_* \sG}}{\Lie{\iota_* \sF}} }$, причём ${ v p^l \pi^{-p^l} \in M_I(\fO_K)[[\Delta]] }$, где ${ l = \left\lfloor \log_p\left( \dfrac{e}{p - 1} \right) \right\rfloor }$.
    \end{enumerate}
\end{guess}

\begin{comment}

\begin{definition}\label{def:4.1:r(F)}
    Пусть ${ \Lambda_F }$ соответствует формальному групповому закону $F$. \textit{Инвариант дробной части} группового закона $F$ определяется как ${ r(F) := \Lambda_F \mod M_m(R) }$. По определению, ${ r(F) \in M_m(K[[\Delta]]) / M_m(R) }$.
\end{definition}

\begin{claim}\label{claim:4.after_r(F):r(F)_incidence}
    Пусть ${ \Lambda_F = v u^{-1} }$ (теорема \ref{theorem:4.0:Lambda_as_fraction}), тогда
    \begin{equation*}
        r(F) \in M_m(R) u^{-1} / M_m(R).
    \end{equation*}
\end{claim}
\begin{proof}
    Пусть ${ \Lambda_F = v u^{-1} }$, ${ v = (v_{i j}) = p^{-l} \pi^{p^l} (w_{i j}) }$, ${ w_{i j} \in \fO_K[[\Delta]] }$.

    Так как ${ R = \bigcup\limits_{i > 0} p^{-i} \fO_K[[\Delta]] }$, то можно считать, что ${ v_{i j} \in R }$. Следовательно, ${ \Lambda_F = v u^{-1} \in M_m(R) u^{-1} }$.
\end{proof}

\begin{theorem}\label{theorem:4.2:main_theorem_for_r(F)}
    Пусть ${ F = (F_1, ..., F_m) }$, ${ G = (G_1, ..., G_k) }$ - формальные групповые законы над $\fO_K$.
    \begin{enumerate}
        \item Пусть ${ f = (f_i(X))_i }$ - некоторый набор из $k$ формальных степенных рядов, такой что ${ f_i \in \fO_K[[X_1, ..., X_m]] }$, ${ f(X) \equiv A p^s X \mod \deg 2 }$, ${ A \in M_{k \times m}(K) }$, ${ s \in \bZ }$. Тогда $f$ является гомоморфизмом из $F$ в $G$ в том и только том случае, если ${ A r(F) = r(G) \varepsilon }$ для некоторой матрицы ${ \varepsilon \in M_{k \times m}(\bQ_p \otimes \fO_{N, \sigma}[[\Delta]]) }$.
        \item Если ${ m = k }$ и высота групповых законов конечна, то $F$ и $G$ изогенны тогда и только тогда, когда существуют матрицы ${ A \in \GL{m}{K} }$ и ${ \varepsilon \in M_m(\bQ_p \otimes \fO_{N, \sigma}[[\Delta]]) }$ такие, что ${ A r(F) = r(G) \varepsilon }$.
        \item Пусть ${ f : F \rightarrow G }$ - гомоморфизм, такой, что ${ f(X) \equiv A X \mod \deg 2 }$ для некоторой матрицы ${ A \in M_{k \times m}(\fO_K) }$. Тогда $f$ можно представить в виде
        \begin{equation*}
            f(X) =
            \sum_{\substack{
                (G) \\
                i = 1, ..., m \\
                j \geq 0 \\
                l = 1, ..., k
            }}
            \left(
                a_{i j l} X_i^{p^j} e_l
            \right),
        \end{equation*}
        где ${ a_{i j l} \in \fO_K }$, ${ e_l = (0, ..., 1, ..., 0) }$ - базисный вектор длины $k$ с единицей на $l$-й позиции. Для этих $a_{i j l}$ выполняется ${ A r(F) = r(G) B }$, где ${ B_{i l} = \sum\limits_{j \geq 0} \theta(\overline{a_{i j l}}) \Delta^j }$, ${ \theta : \overline{N} \rightarrow N }$ - характер Тейхмюллера.
    \end{enumerate}
\end{theorem}
\begin{proof}
    См. \cite{BondarkoVostokov} 4.2.1 или \cite{BondarkoThesis} 4.2.6, главная теорема о "дробных частях".
\end{proof}

\begin{remark}\label{remark:4.after_main_theorem_for_r(F):varepsilon}
   Так как ${ \bQ_p \otimes_{\bZ_p} \fO_{N, \sigma}[[\Delta]] = \bigcup\limits_{i > 0} p^{-i} \fO_{N, \sigma}[[\Delta]] }$, то можно считать, что $\varepsilon$ из теоремы \ref{theorem:4.2:main_theorem_for_r(F)} имеет вид ${ \varepsilon = p^{-i} \delta }$, ${ i > 0 }$, ${ \delta \in M_{k \times m}(\fO_{N, \sigma}[[\Delta]]) }$.
\end{remark}

\begin{remark}\label{remark:4.after_main_theorem_for_r(F):(G)}
   $\sum\limits_{(G)}$ означает, что при суммировании используется операция $+_G$, то есть, для любых допустимых $i_1$, $i_2$, $j_1$, $j_2$, $l_1$, $l_2$ выполняется
   \begin{equation*}
        \left(
            a_{i_1 j_1 l_1} X_{i_1}^{p^{j_1}} e_{l_1}
        \right) +_G
        \left(
            a_{i_2 j_2 l_2} X_{i_2}^{p^{j_2}} e_{l_2}
        \right) =
        G\left(
            \left(
                a_{i_1 j_1 l_1} X_{i_1}^{p^{j_1}} e_{l_1}
            \right),
            \left(
                a_{i_2 j_2 l_2} X_{i_2}^{p^{j_2}} e_{l_2}
            \right)
        \right).
   \end{equation*}
\end{remark}

\end{comment}

\end{section}

\newpage

\printbibliography[title={Список литературы}]

\end{document}